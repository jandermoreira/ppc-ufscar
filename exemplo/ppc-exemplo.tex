%! Author = Ghost Writer
%! Date = 25/04/2025

\documentclass[a3]{ppc-ufscar}

\usepackage{kantlipsum}

\PPCConfig{
    ficha descritiva = false,
}

\PPCDefinaFichaDescritiva{
    instituição = {Instituição de ensino}{Universidade Federal de São Carlos},
    campus = {Campus}{São Carlos},
    centro = {Centro}{Centro de Ciências Aleatórias},
    curso = {Nome do curso}{Bacharelado em Alguma Coisa},
    sigla = {Sigla do curso}{AC},
    modalidade = {Modalidade}{Presencial/Híbrido},
    vagas = {Quantidade de vagas}{60 vagas anuais},
    turno = {Turno}{Integral},
    carga horária = {Carga horária total}{3240 horas},
    regime = {Regime}{Semestral},
    duração = {Tempo de duração}{8 semestres},
    duração mínima = {Tempo de duração mínima}{6 semestres},
    duração máxima = {Tempo de duração máxima}{14 semestres},
    ato legal = {Ato legal}{Parecer 1522/79 de 11 de novembro de 1979},
    última reformulação = {Ano da última reformulação}{2018},
    ano = {Ano}{2025},
    local = {Local}{São Carlos -- SP},
}

\PPCDefinaDadosInstitucionais{
    unidade = {\PPCInfo{instituição}},
    cargo = {Reitora},
    pessoa = {Profa. Dra. Zargulinda Mevânia},
    cargo = {Vice-Reitor},
    pessoa = {Prof. Dr. Quiríntio Marcus Bolvéria},
    cargo = {Pró-Reitor de Graduação},
    pessoa = {Prof. Dr. Tarsílio Vondrake},
    unidade = {\PPCInfo{centro}},
    cargo = {Diretor},
    pessoa = {Prof. Dr. Elvírio Crantobel},
    cargo = {Vice-Diretor},
    pessoa = {Prof. Dr. Mandrílio Estévano},
    unidade = {Coordenação de Curso de \PPCInfo{curso}},
    cargo = {Coordenadora},
    pessoa = {Profa. Dra. Ninfária Clotilde},
    cargo = {Vice-Coordenador},
    pessoa = {Prof. Dr. Valtrônio Mirgélio},
    cargo = {Secretária},
    pessoa = {Sra. Dulmira Xantofreda},
    nova página,
    unidade = {Núcleo Docente Estruturante},
    cargo = {Presidente},
    pessoa = {Prof. Dr. Grivaldo Ortênsio},
    cargo = {Membros},
    pessoa = {Prof. Dr. Jandríaco Fluvério},
    pessoa = {Profa. Dra. Minésia Cravélia},
    pessoa = {Prof. Dr. Odirlânio Tebaldo},
    pessoa = {Profa. Dra. Flandina Quelúria},
    unidade = {Outros colaboradores},
    pessoa = {Sra. Vólmira Trevíssima},
}

\PPCImporteDisciplinas{
    astro-fic, hist-alt-ext, ling-fic, tec-fic,
    bio-sint, antrop-fic,
    geo-fic, med-fic-ext, robot-fic, eco-fic, mus-fic-avanc,
    arqueo-fic, psico-fic-avanc, quim-fic, tec-espec,
    mus-fic, filo-fic,
    lingua-fic, art-fic, med-fic, eco-fic-ext, geo-fic-ext,
    astro-fic-ext, psico-fic, hist-alter, quim-fic-ext,
    tec-fic-avanc, nano-art, arqueo-fic-ext,
    eco-fic-avanc, med-fic-ext2, robot-fic-avanc, tec-fic-ext,
    astro-fic-avanc, hist-alt-avanc, tcc,
}

\PPCDefinaMatrizCurricular{
    nome período = {trimestre},
    número períodos = 8,
    área = {nat-bio}{
        nome = {Fundamentos de Ciências Naturais e Biológicas Fantásticas},
        1 = {astro-fic, bio-sint},
        2 = {eco-fic-avanc, eco-fic-ext},
        3 = {quim-fic},
        5 = {med-fic-ext2, med-fic-ext, eco-fic},
        6 = {med-fic, *quim-fic-ext},
        8 = {astro-fic-avanc, *astro-fic-ext},
    },
    área = {hum-soc}{
        nome = {Fundamentos de Ciências Humanas e Sociais Fantásticas},
        1 = {antrop-fic, arqueo-fic-ext},
        2 = {arqueo-fic},
        3 = {filo-fic, *geo-fic-ext},
        4 = {*hist-alt-avanc, *tec-espec},
        5 = {hist-alt-ext},
        6 = {*ling-fic},
        7 = {lingua-fic, psico-fic-avanc},
        8 = {psico-fic},
    },
    área = {art-ling}{
        nome = {Fundamentos de Artes e Linguagens Fantásticas},
        6 = {*art-fic, *mus-fic-avanc},
        4 = {tec-fic-avanc},
        8 = {mus-fic, nano-art},
    },
    área = {tec-eng}{
        nome = {Fundamentos de Tecnologias e Engenharia Fantásticas},
        3 = {*robot-fic-avanc, *robot-fic},
        7 = {tec-fic-ext, tec-fic, tcc},
    },
    carga optativas = {
        3 = {60, 60},
        4 = {60},
        6 = {30, 60, 60, 60, 120},
    },
    eletivas = {hist-alter, geo-fic},
    carga a compensar = {
        3 = 30,
        8 = 360,
    }
}

% Texto automático para este exemplo
\usepackage{lipsum}  % remova esse pacote na versão real do PPC


%%%%%%%%%%%%%%%%%%%%%%%%%%%%%%%%%%%%%%%%%%%%%%%%%%%%%%%%%%%%%%%%%%%%%%%%%%%%%%%%%%%%%%%
\begin{document}

Centro: \PPCInfo{centro}


\chapter{Ficha descritiva}

\PPCFichaDescritiva


\chapter{Lista de disciplinas}

\subsubsection*{Lista geral}
\begin{itemize}
    \item Start
    \ATForEach{\Disc}{astro-fic-avanc, hist-alt-ext, ling-fic, tec-fic,
        bio-sint, antrop-fic,
        geo-fic, med-fic-ext, robot-fic, eco-fic, mus-fic-avanc,
        arqueo-fic, psico-fic-avanc, quim-fic, tec-espec,
        mus-fic, filo-fic,
        lingua-fic, art-fic, med-fic, eco-fic-ext, geo-fic-ext,
        astro-fic-ext, psico-fic, hist-alter, quim-fic-ext,
        tec-fic-avanc, nano-art, arqueo-fic-ext,
        eco-fic-avanc, med-fic-ext2, robot-fic-avanc, tec-fic-ext,
        astro-fic, hist-alt-avanc
    }{
        \item \textsc{\Disc}: \PPCDisciplina{\Disc}{nome} \ATAttributeIfExist[\Disc]{caráter}{(\PPCDisciplina{\Disc}{caráter})}{***}\par
    }
\end{itemize}

\subsubsection*{Lista por períodos}


\PPCParaCadaPeriodo{\p}{
    \noindent\p --
    \ATListForEach{matriz-principal-periodo-\p}{\Disc}[{ e }{, }{ e }{}]{(\PPCDisciplina{\Disc}{nome})}\par
}


\chapter{Lista de disciplinas por área}

\PPCParaCadaPeriodo{\p}

\ATListForEach{matriz-principal-áreas}{\Area}{
    \subsubsection*{\Area: \PPCArea{\Area}{nome}}
    \PPCParaCadaPeriodo{\p}{
        \noindent\p --~\ATListForEach{matriz-principal-\Area-\p}{\Disc}[{ e }{, }{ e também }{}]{(\Disc)/\PPCDisciplina{\Disc}{nome}}\par
    }
}


\chapter{Carga de optativas}


\PPCParaCadaPeriodo[matriz-principal]{\p}{
    \noindent\p --
    \ATListForEach{matriz-principal-carga-optativas-\p}{\v}[{ e }{, }{ e }{}]{\v}\par
}


\chapter{Carga horária para compensação}

\ExplSyntaxOn
\PPCParaCadaPeriodo{\p}{
    \noindent\p~--~
    \int_use:c {g_ppc_matriz-principal_carga_a_compensar_ \p _int}~horas\par
}
\ExplSyntaxOff


\chapter{Horas}

Cargas horárias:
\begin{itemize}
    \item Total obrigatória: \PPCCargaHoraria{obrigatória}h.
    \item Total optativa: \PPCCargaHoraria{optativa}h.
    \item[+] Total do curso: \PPCCargaHoraria{total}h.
\end{itemize}

Total disponível em eletivas: \PPCCargaHoraria{eletiva}h.

\PPCParaCadaPeriodo{\p}{
    \noindent\textbf{Período: \p}
    \begin{itemize}
        \item Total obrigatória: \PPCCargaHoraria{obrigatória \p}[h].
        \item Total optativa: \PPCCargaHoraria{optativa \p}h.
        \item[+] Total do período: \PPCCargaHoraria{total \p}h.
    \end{itemize}
}


\section{Disciplinas}

\ATForEach{\Id}{
    astro-fic, hist-alt-ext, ling-fic, tec-fic,
    bio-sint, antrop-fic,
    geo-fic, med-fic-ext, robot-fic, eco-fic, mus-fic-avanc,
    arqueo-fic, psico-fic-avanc, quim-fic, tec-espec,
    mus-fic, filo-fic,
    lingua-fic, art-fic, med-fic, eco-fic-ext, geo-fic-ext,
    astro-fic-ext, psico-fic, hist-alter, quim-fic-ext,
    tec-fic-avanc, nano-art, arqueo-fic-ext,
    eco-fic-avanc, med-fic-ext2, robot-fic-avanc, tec-fic-ext,
    astro-fic-avanc, hist-alt-avanc, tcc
}{
    \PPCFichaDisciplina{\Id}%\clearpage
}


\chapter{Desenhos}

\PPCDefinaMatrizCurricular[trilha1]{
    nome período = {semestre},
    número períodos = 8,
    área = {nat-bio}{
        nome = {Trilha da Natureza},
        5 = {med-fic-ext2, med-fic-ext, *eco-fic},
        6 = {med-fic, *quim-fic-ext},
        8 = {astro-fic-avanc, *astro-fic-ext},
    },
}

\begin{landscape}
    % \PPCMatrizCurricular

    \PPCMatrizCurricular<visão do aluno>

    % \PPCMatrizCurricular[trilha1]<>
\end{landscape}


\chapter{Períodos}
\begin{landscape}
    \PPCQuadroPeriodo{}
\end{landscape}


\chapter{Competências}
\begin{itemize}
    \ATListForEach{ppc-competências}{\comp}{
        \item \comp \\
            \PPCCompetencia{\comp}{nome}: \PPCCompetencia{\comp}{descrição}
        \begin{itemize}
            \ATListForEach{ppc-competência-cg-ufscar-aprender}{\espec}{
                \item \espec: \PPCCompetencia{\espec}{nome}/\PPCCompetencia{\espec}{descrição}
            }
        \end{itemize}
    }
\end{itemize}

\end{document}
