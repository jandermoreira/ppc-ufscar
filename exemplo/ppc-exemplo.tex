%! Author = Ghost Writer
%! Date = 25/04/2025

\documentclass[a3]{ppc-ufscar}

\usepackage{ipsum-ptbr}

\PPCConfig{
    ficha descritiva = false,
}

\PPCDefinaFichaDescritiva{
    instituição = {Instituição de ensino}{Universidade Federal de São Carlos},
    campus = {Campus}{São Carlos},
    centro = {Centro}{Centro de Ciências Antigas e Modernas},
    curso = {Nome do curso}{Bacharelado em Saber e Ciências},
    sigla = {Sigla do curso}{AC},
    modalidade = {Modalidade}{Presencial/Híbrido},
    vagas = {Quantidade de vagas}{60 vagas anuais},
    turno = {Turno}{Integral},
    carga horária = {Carga horária total}{\PPCCargaHoraria{total}[\,horas]},
    regime = {Regime}{Semestral},
    duração = {Tempo de duração}{8 semestres},
    duração mínima = {Tempo de duração mínima}{6 semestres},
    duração máxima = {Tempo de duração máxima}{14 semestres},
    ato legal = {Ato legal}{Parecer 1522/79 de 11 de novembro de 1979},
    última reformulação = {Ano da última reformulação}{2018},
    ano = {Ano}{2025},
    local = {Local}{São Carlos -- SP},
}

\PPCDefinaDadosInstitucionais{
    unidade = {\PPCInfo{instituição}},
    cargo = {Reitora},
    pessoa = {Profa. Dra. Zargulinda Mevânia},
    cargo = {Vice-Reitor},
    pessoa = {Prof. Dr. Quiríntio Marcus Bolvéria},
    cargo = {Pró-Reitor de Graduação},
    pessoa = {Prof. Dr. Tarsílio Vondrake},
    unidade = {\PPCInfo{centro}},
    cargo = {Diretor},
    pessoa = {Prof. Dr. Elvírio Crantobel},
    cargo = {Vice-Diretor},
    pessoa = {Prof. Dr. Mandrílio Estévano},
    unidade = {Coordenação de Curso de \PPCInfo{curso}},
    cargo = {Coordenadora},
    pessoa = {Profa. Dra. Ninfária Clotilde},
    cargo = {Vice-Coordenador},
    pessoa = {Prof. Dr. Valtrônio Mirgélio},
    cargo = {Secretária},
    pessoa = {Sra. Dulmira Xantofreda},
    nova página,
    unidade = {Núcleo Docente Estruturante},
    cargo = {Presidente},
    pessoa = {Prof. Dr. Grivaldo Ortênsio},
    cargo = {Membros},
    pessoa = {Prof. Dr. Jandríaco Fluvério},
    pessoa = {Profa. Dra. Minésia Cravélia},
    pessoa = {Prof. Dr. Odirlânio Tebaldo},
    pessoa = {Profa. Dra. Flandina Quelúria},
    unidade = {Outros colaboradores},
    pessoa = {Sra. Vólmira Trevíssima},
}

\PPCImporteAtividades{
    lin-lit-class,
    fil-cos-ant,
    ct-ant,
    hist-mit,
    ret-logica,
    saber-aplic,
    integr,
}

\PPCDefinaMatrizCurricular{
    nome período = {semestre},
    número períodos = 8,
    área = {lin-lit-class}{
        nome = {Línguas e Literatura Clássica},
        1 = {lin-grego, lin-latim},
        2 = {lin-trad, lit-grega},
        3 = {lin-ingant, lit-latina},
        4 = {lin-esttex, *lin-poetrad},
    },
    área = {fil-cos-ant}{
        nome = {Filosofia e Cosmologia Antigas},
        1 = {fil-pre},
        2 = {fil-platao},
        3 = {fil-aristo},
        6 = {*fil-sagrado},
        7 = {fil-neoplat, *fil-helen, *fil-antimod},
    },
    área = {ct-ant}{
        nome = {Ciência e Técnica no Mundo Antigo},
        2 = {ct-cosm, ct-tecn},
        3 = {ct-astr, ct-med},
        5 = {*ct-arq, *ct-alq, *ct-eco},
    },
    área = {hist-mit}{
        nome = {História e Mitologia Comparada},
        1 = {hm-mito, hm-hist},
        2 = {hm-her, hm-rito},
        4 = {*hm-comp},
        5 = {hm-mitmedren},
        7 = {*hm-mod},
    },
% área = {ret-logica}{
%     nome = {Retórica e Lógica},
%     1 = {rl-intro},
%     2 = {rl-helen},
%     3 = {rl-medieval},
%     4 = {rl-mod},
%     5 = {rl-contemp},
%     6 = {*rl-etica, *rl-logica},
%     7 = {*rl-estetica},
% },
    área = {saber-aplic}{
        nome = {Transmissão do Saber e Aplicações Contemporâneas},
        4 = {ts-preserv},
        5 = {ts-humanismo},
        6 = {*ts-alexandria},
        7 = {ts-modernidade}%, *ts-humanidades},
    },
    área = {integr}{
        nome = {Atividades Integradoras (Pesquisa, Extensão e TCC)},
        1 = {ai-leitura},
        3 = {ai-labcultura},
        4 = {ai-proj1},
        6 = {ai-proj2, *ai-producao},
        7 = {*ai-metodologia, ai-seminario},
        8 = {-ai-tcc, -ai-estagio, +180},
    },
    carga optativas = {
        4 = {60},
        5 = {60, 60},
        6 = {60, 60, 60},
        7 = {60, 60, 60, 60}
    },
% eletivas = {},
}


%%%%%%%%%%%%%%%%%%%%%%%%%%%%%%%%%%%%%%%%%%%%%%%%%%%%%%%%%%%%%%%%%%%%%%%%%%%%%%%%%%%%%%%
\begin{document}

\chapter{Apresentação}

\ptbrsentenca[201-202]\par
\ptbrparagrafo[1-3]


\chapter{Ficha descritiva}
\ptbrsentenca[18-25]

A ficha descritiva é apresentada no \Cref{quadro:ficha-descritiva}.

\begin{quadro}
    \PPCFichaDescritiva
    \caption{Ficha descritiva do \PPCInfo{curso}.}
    \label{quadro:ficha-descritiva}
\end{quadro}

\ptbrparagrafo[60]


\chapter{Histórico do Curso de \PPCInfo{curso}}
\ptbrparagrafo[40]


\section{Implantação}
\ptbrparagrafo[31-35]


\section{Situação atual}
\ptbrparagrafo[38-41]


\chapter{Marco Referencial}
\ptbrparagrafo[10-12]


\chapter{Marco Conceitual}
\ptbrparagrafo[21-24]


\chapter{Marco Estrutural}
\ptbrsentenca[61-62]


\section{Visão Geral}
\ptbrparagrafo[15-16]


\section{Matriz Curricular}
\ptbrparagrafo[45]

A matriz curricular com as atividades curriculares obrigatórias e optativas está representada no \Cref{quadro:matriz-completa}. A matriz curricular, pela visão do aluno, está apresentada no \Cref{quadro:matriz-aluno}

\begin{landscape}
    \begin{quadro}
        \caption{Matriz curricular de atividades curriculares do \PPCInfo{curso}.}
        \label{quadro:matriz-completa}
        \begin{center}
            \PPCMatrizCurricular
        \end{center}
    \end{quadro}
\end{landscape}

\begin{quadro}
    \caption{Matriz curricular de atividades curriculares na visão do aluno.}
    \label{quadro:matriz-aluno}
    \begin{center}
        \PPCMatrizCurricular<visão do aluno>
    \end{center}
\end{quadro}

\ptbrparagrafo[70]


\section{Trilhas}

O curso define várias trilhas. \ptbrparagrafo[70].

Uma trilha pode ser definida no curso, como apresentado no \Cref{quadro:trilha}.


\PPCDefinaMatrizCurricular[trilha]{
    nome período = {semestre},
    número períodos = 8,
    área = {saberes}{
        nome = {Trilha dos Saberes},
        1 = {hm-mito, hm-hist},
        2 = {hm-her, hm-rito},
        4 = {*hm-comp},
        5 = {hm-mitmedren},
    },
    carga optativas = {
        5 = {60},
        6 = {60},
        8 = {60},
    }
}

\begin{quadro}
    \caption{Trilha especial no curso (com certificação).}
    \label{quadro:trilha}
    \begin{center}
        \PPCMatrizCurricular[trilha]
    \end{center}
\end{quadro}

\ptbrsentenca[700-703]


\section{Ementário}
As atividades curriculares do curso estão aqui. \ptbrparagrafo[11-12]

\ATListForEach{matriz-principal-atividades}{\identificador}{
    \PPCFichaAtividade{\identificador}
}

Exemplo separando por períodos
\PPCParaCadaPeriodo{\periodo}{
    \subsection{Ementas das atividades curriculares de \periodoº \ATAttributeGet[matriz-principal]{nome período}}
    \ATListForEach{matriz-principal-periodo-\periodo}{\identificador}{
        \PPCFichaAtividade{\identificador}
    }
}


\section{Quadro de atividades curriculares}
Seguem os quadros de atividades curriculares. \ptbrsentenca[66]

\begin{landscape}
    \subsection{Atividades da primeira metade do curso}

    \PPCQuadroPeriodo{1, 2, 3, 4}

    \subsection{Atividades da segunda metade do curso}

    \PPCQuadroPeriodo{5, 6, 7, 8}
\end{landscape}


\appendix


\chapter{Resumo das Informações do Curso}


\section{Lista de Atividades por Período}

\begin{itemize}
    \PPCParaCadaPeriodo{\p}{
        \subsection*{\pº Período}
        \ATListForEach{matriz-principal-periodo-\p}{\Disc}[{ e }{; }{; e }{}]{\item \PPCAtividade{\Disc}{nome}}
    }
\end{itemize}


% \section{Lista de Atividades por Área}
%
% \ATListForEach{matriz-principal-áreas}{\Area}{
%     \subsection*{\PPCArea{\Area}{nome}}
%     \begin{itemize}
%         \PPCParaCadaPeriodo{\p}{
%                 \item b
%             \ATListForEach{matriz-principal-\Area-\p}{\Disc}[{; }{; }{; }{.}]{\item  \PPCAtividade{\Disc}{nome}}
%         }
%     \end{itemize}
% }


\section{Cargas Horárias}
Cargas horárias envolvidas.

\begin{itemize}
    \item Cargas horárias totais
    \begin{itemize}
        \item Total obrigatória: \PPCCargaHoraria{obrigatória}[\,h].
        \item Total optativa: \PPCCargaHoraria{optativa}[\,h].
        \item[+] Total do curso: \PPCCargaHoraria{total}[\,h].
    \end{itemize}
    % \item Total disponível em eletivas: \PPCCargaHoraria{eletiva}h.

    \PPCParaCadaPeriodo{\p}{
        \item Carga no período: \pº
        \begin{itemize}
            \item Total obrigatória: \PPCCargaHoraria{obrigatória \p}[\,h].
            \item Total optativa: \PPCCargaHoraria{optativa \p}[\,h].
            \item[+] Total do período: \PPCCargaHoraria{total \p}[\,h].
        \end{itemize}
    }
\end{itemize}

\end{document}
