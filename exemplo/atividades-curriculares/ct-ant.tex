\PPCDefinaAtividade{ct-cosm}{
    código = {1303001},
    nome = {Cosmologia e Matemática Antiga},
    objetivo = {Compreender as concepções de cosmos e número nas civilizações antigas, relacionando saber astronômico e filosofia natural.},
    ementa = {Origens da cosmologia matemática. Pitágoras e a harmonia do mundo. Geometria e medida no Egito e na Grécia. A noção de ordem cósmica e sua relação com o pensamento filosófico e religioso. Exercícios de reconstrução geométrica e simbólica.},
    pré-requisitos = {},
    horas teóricas = 35,
    horas práticas = 25,
    horas extensionistas = 0,
    horas estágio = 0,
    departamento = DF,
    bibliografia básica = {Silva, E. \textit{O Número e o Cosmos}. Areté, 2018.
    \newline Mendes, C. \textit{Geometria Sagrada na Antiguidade}. Polis, 2019.
    \newline Duarte, M. \textit{Matemática e Mistério}. Hélios, 2020.
    \newline Oliveira, R. \textit{As Esferas do Mundo}. Arqué, 2017.},
    bibliografia complementar = {Brito, A. \textit{Os Céus e o Cálculo Antigo}. Herma, 2018.
    \newline Farias, V. \textit{Cosmologia Simbólica}. Thalassa, 2020.
    \newline Klein, F. \textit{A Ordem e o Infinito}. Aurora, 2021.},
    competências = {Relacionar matemática e cosmologia antigas; compreender a dimensão simbólica do número; interpretar modelos de ordem cósmica.},
}

\PPCDefinaAtividade{ct-tecn}{
    código = {1303002},
    nome = {Técnica e Invenção no Mundo Antigo},
    objetivo = {Analisar o desenvolvimento técnico nas civilizações antigas e suas implicações filosóficas e culturais.},
    ementa = {Instrumentos e engenhos da Antiguidade. Arquitetura, mecânica e hidráulica antigas. O papel do artesão e do engenheiro. A técnica como extensão do corpo e expressão do logos prático. Estudo de artefatos e reconstruções experimentais.},
    pré-requisitos = {},
    horas teóricas = 30,
    horas práticas = 30,
    horas extensionistas = 0,
    horas estágio = 0,
    departamento = DT,
    bibliografia básica = {Oliveira, R. \textit{A Mão e o Mundo}. Polis, 2019.
    \newline Duarte, M. \textit{Engenhos e Imaginação Técnica}. Areté, 2020.
    \newline Mendes, C. \textit{A Arte dos Antigos Mestres}. Hélios, 2018.
    \newline Lemos, P. \textit{Tecnologia e Sabedoria Antiga}. Arqué, 2021.},
    bibliografia complementar = {Brito, A. \textit{Máquinas e Mitos}. Herma, 2019.
    \newline Farias, V. \textit{A Técnica como Filosofia}. Aurora, 2020.
    \newline Klein, F. \textit{O Saber Manual}. Thalassa, 2018.},
    competências = {Reconhecer a dimensão cultural e simbólica da técnica; compreender a relação entre invenção e conhecimento; valorizar o saber técnico como forma de pensamento.},
}

\PPCDefinaAtividade{ct-astr}{
    código = {1303003},
    nome = {Astronomia Antiga e Observação do Céu},
    objetivo = {Estudar o desenvolvimento da astronomia antiga e suas relações com a filosofia e o mito.},
    ementa = {Observação do céu a olho nu. Astronomia babilônica, egípcia e grega. O modelo geocêntrico e suas interpretações filosóficas. A astronomia como fundamento da ordem e do tempo. Práticas de observação e registro simbólico.},
    pré-requisitos = {ct-cosm},
    horas teóricas = 35,
    horas práticas = 15,
    horas extensionistas = 10,
    horas estágio = 0,
    departamento = DF,
    bibliografia básica = {Silva, E. \textit{Os Céus dos Antigos}. Hélios, 2018.
    \newline Oliveira, R. \textit{A Harmonia das Estrelas}. Arqué, 2019.
    \newline Mendes, C. \textit{Astronomia e Sabedoria Antiga}. Areté, 2021.
    \newline Duarte, M. \textit{O Olhar e o Cosmos}. Polis, 2017.},
    bibliografia complementar = {Brito, A. \textit{Calendários e Cosmos}. Herma, 2019.
    \newline Farias, V. \textit{Astrologia e Filosofia}. Aurora, 2020.
    \newline Klein, F. \textit{Tempo e Ordem Celeste}. Thalassa, 2021.},
    competências = {Interpretar os fundamentos da astronomia antiga; reconhecer a relação entre observação e simbolismo; integrar experiência sensível e racionalidade cósmica.},
}

\PPCDefinaAtividade{ct-med}{
    código = {1303004},
    nome = {Medicina, Corpo e Natureza na Antiguidade},
    objetivo = {Compreender as concepções antigas de saúde, corpo e equilíbrio, articulando saber médico, filosofia e religião.},
    ementa = {Teorias dos quatro humores. Hipócrates e Galeno. Medicina sagrada e racional. Corpo, alma e cosmos. O cuidado de si e o ideal de harmonia natural. Práticas interpretativas de textos médicos antigos.},
    pré-requisitos = {},
    horas teóricas = 40,
    horas práticas = 20,
    horas extensionistas = 0,
    horas estágio = 0,
    departamento = DB,
    bibliografia básica = {Mendes, C. \textit{O Corpo e o Cosmos}. Areté, 2019.
    \newline Duarte, M. \textit{Hipócrates e o Equilíbrio}. Hélios, 2018.
    \newline Oliveira, R. \textit{Medicina Antiga e Filosofia}. Arqué, 2021.
    \newline Silva, E. \textit{A Natureza e o Cuidado}. Polis, 2020.},
    bibliografia complementar = {Brito, A. \textit{Corpo e Sabedoria Antiga}. Herma, 2019.
    \newline Farias, V. \textit{Entre Cura e Cosmos}. Thalassa, 2020.
    \newline Klein, F. \textit{O Corpo e o Tempo}. Aurora, 2018.},
    competências = {Compreender a visão holística do corpo na Antiguidade; relacionar medicina e filosofia; desenvolver reflexão ética sobre o cuidado e a natureza.},
}

\PPCDefinaAtividade{ct-arq}{
    código = {1303005},
    nome = {Arquitetura e Simbolismo na Antiguidade},
    objetivo = {Investigar a arquitetura antiga como expressão do pensamento cosmológico e técnico, integrando arte, matemática e religião.},
    ementa = {Proporções sagradas e harmonia arquitetônica. Templos, cidades e monumentos como representações do cosmos. Técnicas construtivas e significado simbólico. Estudos de casos e maquetes conceituais.},
    pré-requisitos = {},
    horas teóricas = 30,
    horas práticas = 20,
    horas extensionistas = 10,
    horas estágio = 0,
    departamento = DT,
    bibliografia básica = {Oliveira, R. \textit{Arquitetura e Cosmos}. Polis, 2019.
    \newline Mendes, C. \textit{As Formas do Sagrado}. Areté, 2020.
    \newline Duarte, M. \textit{Construções e Simbolismo}. Arqué, 2018.
    \newline Lemos, P. \textit{Medida e Mistério}. Hélios, 2021.},
    bibliografia complementar = {Brito, A. \textit{Espaço e Ritual}. Herma, 2019.
    \newline Farias, V. \textit{A Geometria do Sagrado}. Thalassa, 2020.
    \newline Klein, F. \textit{Arquitetura e Filosofia Antiga}. Aurora, 2018.},
    competências = {Reconhecer a dimensão simbólica da arquitetura antiga; relacionar técnica, arte e cosmologia; desenvolver leitura espacial e filosófica das formas.},
}

\PPCDefinaAtividade{ct-alq}{
    código = {1303006},
    nome = {Alquimia e Filosofia Natural},
    objetivo = {Analisar a alquimia como forma de conhecimento simbólico e experimental, antecedente das ciências naturais modernas.},
    ementa = {Origens e tradições alquímicas. Elementos, metais e transformações. A relação entre matéria e espírito. Interpretação filosófica e mística dos processos químicos antigos. Oficinas simbólicas e leitura de tratados.},
    pré-requisitos = {ct-cosm},
    horas teóricas = 35,
    horas práticas = 15,
    horas extensionistas = 10,
    horas estágio = 0,
    departamento = DB,
    bibliografia básica = {Silva, E. \textit{A Arte da Transformação}. Hélios, 2019.
    \newline Mendes, C. \textit{Matéria e Espírito}. Areté, 2021.
    \newline Duarte, M. \textit{Alquimia e Filosofia Natural}. Arqué, 2020.
    \newline Oliveira, R. \textit{Os Elementos e o Ser}. Polis, 2018.},
    bibliografia complementar = {Brito, A. \textit{Símbolos e Metais}. Herma, 2019.
    \newline Farias, V. \textit{A Ciência dos Antigos}. Aurora, 2020.
    \newline Klein, F. \textit{O Fogo e o Pensamento}. Thalassa, 2021.},
    competências = {Compreender o significado filosófico da alquimia; relacionar experimentação e simbolismo; identificar raízes antigas das ciências naturais.},
}

\PPCDefinaAtividade{ct-eco}{
    código = {1303007},
    nome = {Ecologia e Saberes da Natureza Antiga},
    objetivo = {Investigar as concepções ecológicas implícitas na filosofia e nas ciências naturais antigas, enfatizando o equilíbrio entre ser humano e ambiente.},
    ementa = {Natureza como organismo vivo. A ética da terra nos filósofos antigos. Relação entre cosmos, planta e animal. O ideal de harmonia e reciprocidade. Atividades de campo e análise comparativa de textos antigos e contemporâneos.},
    pré-requisitos = {ct-med},
    horas teóricas = 30,
    horas práticas = 20,
    horas extensionistas = 10,
    horas estágio = 0,
    departamento = DB,
    bibliografia básica = {Mendes, C. \textit{A Natureza Viva}. Areté, 2020.
    \newline Oliveira, R. \textit{O Equilíbrio dos Seres}. Polis, 2019.
    \newline Duarte, M. \textit{Filosofia e Ecologia Antiga}. Arqué, 2018.
    \newline Lemos, P. \textit{O Mundo como Jardim}. Hélios, 2021.},
    bibliografia complementar = {Brito, A. \textit{A Terra e o Ser}. Herma, 2020.
    \newline Farias, V. \textit{Ecologia e Cosmologia}. Thalassa, 2019.
    \newline Klein, F. \textit{Saberes do Verde}. Aurora, 2021.},
    competências = {Reconhecer princípios ecológicos na filosofia antiga; analisar concepções de natureza e reciprocidade; integrar saber antigo e consciência ambiental contemporânea.},
}
