\PPCDefinaAtividade{hm-mito}{
    código = {1404001},
    nome = {Mitologia Clássica e Narrativas da Origem},
    objetivo = {Compreender os principais mitos da Antiguidade e suas funções simbólicas, sociais e filosóficas na formação do pensamento ocidental.},
    ementa = {Cosmogonias e genealogias divinas. Mitologia grega, romana e oriental. O mito como forma de pensamento e narrativa fundadora. Interpretação simbólica, filosófica e literária dos mitos da criação e da ordem.},
    pré-requisitos = {},
    horas teóricas = 40,
    horas práticas = 15,
    horas extensionistas = 5,
    horas estágio = 0,
    departamento = DH,
    bibliografia básica = {Silva, E. \textit{Os Deuses e as Origens}. Areté, 2018.
    \newline Mendes, C. \textit{Mito e Razão}. Hélios, 2019.
    \newline Duarte, M. \textit{As Narrativas do Começo}. Polis, 2020.
    \newline Oliveira, R. \textit{Mitologia e Conhecimento}. Arqué, 2017.},
    bibliografia complementar = {Brito, A. \textit{O Simbolismo do Mito}. Herma, 2018.
    \newline Farias, V. \textit{Deuses e Arquétipos}. Thalassa, 2020.
    \newline Klein, F. \textit{O Nascimento do Mundo}. Aurora, 2019.},
    competências = {Interpretar mitos clássicos e suas funções simbólicas; compreender a mitologia como forma de pensamento; relacionar mito, rito e linguagem.},
}

\PPCDefinaAtividade{hm-her}{
    código = {1404002},
    nome = {Heróis, Deuses e Arquétipos},
    objetivo = {Investigar as figuras heroicas e divinas na mitologia clássica e comparada, explorando sua dimensão simbólica e ética.},
    ementa = {O herói como mediador entre o humano e o divino. Ciclos heroicos e arquétipos universais. Interpretação simbólica e psicológica das jornadas míticas. Relações entre mitologia, arte e filosofia moral.},
    pré-requisitos = {hm-mito},
    horas teóricas = 35,
    horas práticas = 20,
    horas extensionistas = 5,
    horas estágio = 0,
    departamento = DH,
    bibliografia básica = {Oliveira, R. \textit{A Jornada do Herói Antigo}. Areté, 2020.
    \newline Mendes, C. \textit{Arquétipos e Destino}. Hélios, 2018.
    \newline Duarte, M. \textit{Entre Deuses e Homens}. Polis, 2021.
    \newline Silva, E. \textit{Heróis e Metamorfoses}. Arqué, 2019.},
    bibliografia complementar = {Brito, A. \textit{Psicologia do Mito}. Herma, 2019.
    \newline Farias, V. \textit{O Herói e o Sagrado}. Aurora, 2020.
    \newline Klein, F. \textit{Mito, Ética e Imagem}. Thalassa, 2021.},
    competências = {Reconhecer arquétipos e funções do herói; analisar narrativas heroicas; compreender a dimensão ética e simbólica do mito.},
}

\PPCDefinaAtividade{hm-hist}{
    código = {1404003},
    nome = {História Cultural do Mundo Antigo},
    objetivo = {Analisar as estruturas culturais, políticas e simbólicas das civilizações antigas, enfatizando as interações entre mito, rito e poder.},
    ementa = {Sociedades antigas: Grécia, Roma, Egito e Mesopotâmia. Religião, arte e política como sistemas simbólicos. Ritos de passagem e festividades. O papel da tradição oral e da memória coletiva.},
    pré-requisitos = {},
    horas teóricas = 40,
    horas práticas = 15,
    horas extensionistas = 5,
    horas estágio = 0,
    departamento = DH,
    bibliografia básica = {Duarte, M. \textit{Cultura e Poder no Mundo Antigo}. Polis, 2018.
    \newline Mendes, C. \textit{História e Simbolismo}. Areté, 2020.
    \newline Oliveira, R. \textit{Civilizações e Memória}. Arqué, 2019.
    \newline Silva, E. \textit{O Tempo dos Deuses}. Hélios, 2021.},
    bibliografia complementar = {Brito, A. \textit{Sociedade e Ritual}. Herma, 2019.
    \newline Farias, V. \textit{Imaginário Antigo}. Thalassa, 2020.
    \newline Klein, F. \textit{História e Mito}. Aurora, 2018.},
    competências = {Compreender as bases culturais do mundo antigo; relacionar mito, rito e sociedade; desenvolver leitura simbólica da história.},
}

\PPCDefinaAtividade{hm-rito}{
    código = {1404004},
    nome = {Religião, Rito e Sociedade na Antiguidade},
    objetivo = {Estudar as práticas religiosas e rituais das sociedades antigas, analisando suas relações com o espaço social e o poder simbólico.},
    ementa = {Religiões cívicas e mistéricas. O papel do sacerdote e da comunidade. Rituais de purificação, sacrifício e iniciação. Relações entre religião, política e identidade cultural. Observação comparada de ritos e festivais.},
    pré-requisitos = {hm-hist},
    horas teóricas = 35,
    horas práticas = 20,
    horas extensionistas = 5,
    horas estágio = 0,
    departamento = DH,
    bibliografia básica = {Silva, E. \textit{O Sagrado e o Poder}. Areté, 2018.
    \newline Duarte, M. \textit{Ritos e Comunidade}. Polis, 2021.
    \newline Oliveira, R. \textit{As Religiões do Mundo Antigo}. Arqué, 2019.
    \newline Mendes, C. \textit{Mistérios e Sacrifício}. Hélios, 2020.},
    bibliografia complementar = {Brito, A. \textit{A Festa e o Divino}. Herma, 2019.
    \newline Farias, V. \textit{Rituais e Política}. Thalassa, 2021.
    \newline Klein, F. \textit{O Corpo do Sagrado}. Aurora, 2018.},
    competências = {Analisar as práticas religiosas antigas; compreender a relação entre rito e poder social; interpretar o simbolismo do sagrado.},
}

\PPCDefinaAtividade{hm-comp}{
    código = {1404005},
    nome = {Mitologia Comparada e Tradições Simbólicas},
    objetivo = {Explorar as convergências e divergências entre mitologias de diferentes culturas, identificando arquétipos universais e singularidades simbólicas.},
    ementa = {Comparação entre mitos gregos, egípcios, mesopotâmicos, orientais e ameríndios. Estruturas simbólicas e funções narrativas. Interpretações filosóficas, psicológicas e antropológicas. Oficina de leitura comparativa de mitos.},
    pré-requisitos = {hm-her},
    horas teóricas = 30,
    horas práticas = 25,
    horas extensionistas = 5,
    horas estágio = 0,
    departamento = DH,
    bibliografia básica = {Oliveira, R. \textit{O Mito e o Mundo}. Areté, 2019.
    \newline Mendes, C. \textit{Mitologia Comparada}. Hélios, 2018.
    \newline Duarte, M. \textit{Arquétipos do Oriente e do Ocidente}. Polis, 2020.
    \newline Silva, E. \textit{O Espelho dos Deuses}. Arqué, 2021.},
    bibliografia complementar = {Brito, A. \textit{Estruturas do Imaginário}. Herma, 2020.
    \newline Farias, V. \textit{Mitologias Universais}. Thalassa, 2019.
    \newline Klein, F. \textit{Simbologia e Diferença}. Aurora, 2021.},
    competências = {Comparar mitologias e reconhecer arquétipos; interpretar mitos de diferentes culturas; compreender o papel do mito como linguagem simbólica universal.},
}

\PPCDefinaAtividade{hm-mitmedren}{
    código = {1404006},
    nome = {A Mitologia na Idade Média e no Renascimento},
    objetivo = {Analisar as reelaborações e permanências dos mitos clássicos na tradição medieval e renascentista.},
    ementa = {Transformações do imaginário clássico. Mitos reinterpretados por teólogos, artistas e filósofos. A simbologia da luz, do herói e da ordem cósmica na cultura cristã e humanista. Leitura de textos e imagens de recepção clássica.},
    pré-requisitos = {hm-comp},
    horas teóricas = 35,
    horas práticas = 15,
    horas extensionistas = 10,
    horas estágio = 0,
    departamento = DH,
    bibliografia básica = {Mendes, C. \textit{Do Mito ao Humanismo}. Hélios, 2018.
    \newline Duarte, M. \textit{O Legado dos Deuses}. Polis, 2020.
    \newline Oliveira, R. \textit{Renascimento e Simbolismo}. Arqué, 2019.
    \newline Silva, E. \textit{Luz e Harmonia}. Areté, 2021.},
    bibliografia complementar = {Brito, A. \textit{O Retorno dos Deuses}. Herma, 2019.
    \newline Farias, V. \textit{Imaginário Cristão e Clássico}. Thalassa, 2020.
    \newline Klein, F. \textit{Humanismo e Mito}. Aurora, 2021.},
    competências = {Compreender a recepção dos mitos antigos; identificar continuidades simbólicas; relacionar o pensamento clássico ao renascimento cultural europeu.},
}

\PPCDefinaAtividade{hm-mod}{
    código = {1404007},
    nome = {Mito e Modernidade: Releituras Contemporâneas},
    objetivo = {Examinar as recriações modernas e contemporâneas dos mitos clássicos em diferentes linguagens artísticas e filosóficas.},
    ementa = {O mito na literatura, no cinema e nas artes visuais. Releituras filosóficas e psicanalíticas. O herói moderno e o anti-herói. O mito como crítica cultural e forma de resistência simbólica.},
    pré-requisitos = {hm-mitmedren},
    horas teóricas = 30,
    horas práticas = 20,
    horas extensionistas = 10,
    horas estágio = 0,
    departamento = DH,
    bibliografia básica = {Silva, E. \textit{Os Mitos do Presente}. Areté, 2019.
    \newline Mendes, C. \textit{Modernidade e Arquétipo}. Hélios, 2020.
    \newline Duarte, M. \textit{Cinema e Mito}. Polis, 2021.
    \newline Oliveira, R. \textit{Mitopoéticas Contemporâneas}. Arqué, 2018.},
    bibliografia complementar = {Brito, A. \textit{O Mito na Era Digital}. Herma, 2020.
    \newline Farias, V. \textit{Narrativas e Símbolos Modernos}. Thalassa, 2019.
    \newline Klein, F. \textit{O Herói Fragmentado}. Aurora, 2021.},
    competências = {Interpretar releituras modernas do mito; analisar mitos na arte e na cultura contemporânea; refletir sobre o papel simbólico do mito na modernidade.},
}
