\PPCDefinaAtividade{ts-preserv}{
    código = {1606001},
    nome = {Preservação e Tradução do Conhecimento},
    objetivo = {Compreender os processos históricos e técnicos de preservação, tradução e transmissão do saber desde a Antiguidade até o presente.},
    ementa = {História da transmissão do conhecimento. Tradução e conservação de textos antigos. Bibliotecas, escribas e manuscritos. Tecnologias de registro e difusão do saber. Ética da preservação cultural.},
    pré-requisitos = {},
    horas teóricas = 35,
    horas práticas = 15,
    horas extensionistas = 10,
    horas estágio = 0,
    departamento = DH,
    bibliografia básica = {Lima, P. \textit{O Saber e o Tempo}. Arcádia, 2018.
    \newline Rodrigues, T. \textit{Tradução e Memória Cultural}. Lúmen, 2019.
    \newline Castro, A. \textit{Bibliotecas da Antiguidade}. Solaris, 2020.
    \newline Nogueira, L. \textit{Guardadores de Livros}. Ânfora, 2021.},
    bibliografia complementar = {Dias, M. \textit{O Texto e a Tradição}. Arqué, 2020.
    \newline Silva, J. \textit{A Cultura Manuscrita}. Polis, 2018.
    \newline Melo, R. \textit{Memória e Conhecimento}. Thalassa, 2019.},
    competências = {Compreender a importância da tradução e preservação para a continuidade do saber; analisar contextos históricos de transmissão; aplicar práticas de conservação e difusão cultural.},
}

\PPCDefinaAtividade{ts-humanismo}{
    código = {1606002},
    nome = {Humanismo e Ciência na Era Moderna},
    objetivo = {Analisar a relação entre o humanismo renascentista e o surgimento da ciência moderna, destacando os processos de secularização do saber.},
    ementa = {O Renascimento e a redescoberta dos clássicos. Humanismo, arte e filosofia. A nova ciência e o método experimental. Conflitos entre fé, razão e liberdade intelectual.},
    pré-requisitos = {ts-preserv},
    horas teóricas = 40,
    horas práticas = 10,
    horas extensionistas = 10,
    horas estágio = 0,
    departamento = DF,
    bibliografia básica = {Rodrigues, T. \textit{Humanismo e Descoberta}. Lúmen, 2020.
    \newline Lima, P. \textit{O Nascimento da Ciência Moderna}. Arcádia, 2019.
    \newline Mendes, F. \textit{Razão e Experiência}. Polis, 2021.
    \newline Castro, A. \textit{A Revolução do Saber}. Solaris, 2018.},
    bibliografia complementar = {Nogueira, L. \textit{Da Arte à Ciência}. Ânfora, 2020.
    \newline Dias, M. \textit{Humanismo e Modernidade}. Arqué, 2021.
    \newline Melo, R. \textit{O Saber Renascentista}. Thalassa, 2019.},
    competências = {Relacionar o pensamento humanista à formação da ciência moderna; compreender o papel do humanismo na modernidade; analisar transformações na relação entre homem e conhecimento.},
}

\PPCDefinaAtividade{ts-alexandria}{
    código = {1606003},
    nome = {Alexandria e as Bibliotecas do Mundo Antigo},
    objetivo = {Investigar o papel das bibliotecas antigas na formação da cultura escrita e na difusão do conhecimento científico e filosófico.},
    ementa = {A Biblioteca de Alexandria e os centros do saber antigo. Filologia, tradução e ensino na Antiguidade. Organização e circulação dos textos. Incêndios, perdas e reconstruções simbólicas.},
    pré-requisitos = {},
    horas teóricas = 30,
    horas práticas = 20,
    horas extensionistas = 10,
    horas estágio = 0,
    departamento = DH,
    bibliografia básica = {Silva, J. \textit{Alexandria: O Farol do Saber}. Polis, 2019.
    \newline Castro, A. \textit{Bibliotecas Antigas e Conhecimento}. Solaris, 2021.
    \newline Lima, P. \textit{Livros e Labirintos}. Arcádia, 2020.
    \newline Rodrigues, T. \textit{A Palavra e o Papiro}. Lúmen, 2018.},
    bibliografia complementar = {Dias, M. \textit{Os Guardiões do Conhecimento}. Arqué, 2021.
    \newline Melo, R. \textit{Tradições do Texto Antigo}. Thalassa, 2019.
    \newline Nogueira, L. \textit{Memória e Biblioteca}. Ânfora, 2020.},
    competências = {Compreender a importância das bibliotecas antigas; identificar processos de transmissão textual; refletir sobre a preservação do conhecimento.},
}

\PPCDefinaAtividade{ts-modernidade}{
    código = {1606004},
    nome = {Antiguidade e Modernidade: Tradições em Transformação},
    nome abreviado = {Antiguidade e Modernidade},
    objetivo = {Estudar a recepção da Antiguidade na cultura moderna e contemporânea, com ênfase na adaptação de conceitos, mitos e formas de saber.},
    ementa = {Recepção dos clássicos na modernidade. A tradição filosófica antiga na arte, ciência e política modernas. Transformações do mito e da linguagem. O legado clássico na cultura contemporânea.},
    pré-requisitos = {ts-humanismo, ts-alexandria},
    horas teóricas = 35,
    horas práticas = 15,
    horas extensionistas = 10,
    horas estágio = 0,
    departamento = DH,
    bibliografia básica = {Mendes, F. \textit{O Antigo no Novo}. Polis, 2020.
    \newline Lima, P. \textit{Tradição e Ruptura}. Arcádia, 2019.
    \newline Rodrigues, T. \textit{A Herança dos Clássicos}. Lúmen, 2021.
    \newline Nogueira, L. \textit{Modernidades do Passado}. Ânfora, 2018.},
    bibliografia complementar = {Castro, A. \textit{Clássicos e Modernos}. Solaris, 2021.
    \newline Melo, R. \textit{O Tempo da Tradição}. Thalassa, 2020.
    \newline Dias, M. \textit{Memória Cultural e Atualidade}. Arqué, 2019.},
    competências = {Analisar a recepção de ideias antigas em contextos modernos; compreender transformações culturais do legado clássico; aplicar conceitos de tradição e reinterpretação.},
}

\PPCDefinaAtividade{ts-humanidades}{
    código = {1606005},
    nome = {Saberes Antigos e Humanidades Digitais},
    objetivo = {Explorar o diálogo entre estudos clássicos e tecnologias digitais, desenvolvendo competências em pesquisa e difusão do saber no ambiente digital.},
    ementa = {Humanidades digitais e preservação do conhecimento. Ferramentas digitais para estudos clássicos. Edições eletrônicas, bancos de dados e repositórios. Ética, autoria e acesso ao conhecimento.},
    pré-requisitos = {ts-humanismo, ts-modernidade},
    horas teóricas = 30,
    horas práticas = 20,
    horas extensionistas = 10,
    horas estágio = 0,
    departamento = DH,
    bibliografia básica = {Rodrigues, T. \textit{Humanidades Digitais e Memória}. Lúmen, 2021.
    \newline Lima, P. \textit{Tecnologia e Tradição}. Arcádia, 2020.
    \newline Melo, R. \textit{Arquivo e Rede}. Thalassa, 2019.
    \newline Castro, A. \textit{O Saber Digital}. Solaris, 2021.},
    bibliografia complementar = {Dias, M. \textit{Humanismo e Tecnologia}. Arqué, 2019.
    \newline Nogueira, L. \textit{Preservar no Ciberespaço}. Ânfora, 2020.
    \newline Silva, J. \textit{Cultura e Algoritmo}. Polis, 2021.},
    competências = {Aplicar recursos digitais à pesquisa em humanidades; compreender desafios contemporâneos da preservação e difusão; integrar saber clássico e inovação tecnológica.},
}
