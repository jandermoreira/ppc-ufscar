\PPCDefinaAtividade{rl-intro}{
    código = {1505001},
    nome = {Introdução à Filosofia Antiga},
    objetivo = {Apresentar as origens do pensamento filosófico ocidental na Grécia Antiga, enfatizando os problemas cosmológicos, ontológicos e éticos.},
    ementa = {O nascimento da filosofia na Jônia. Os filósofos pré-socráticos e a questão do princípio (arché). Sócrates e a virada ética. Platão e o mundo das ideias. Aristóteles e a sistematização do saber.},
    pré-requisitos = {},
    horas teóricas = 40,
    horas práticas = 10,
    horas extensionistas = 0,
    horas estágio = 0,
    departamento = DF,
    bibliografia básica = {Oliveira, R. \textit{Filosofia e Cosmos}. Areté, 2018.
    \newline Mendes, C. \textit{Platão e o Mundo das Ideias}. Hélios, 2020.
    \newline Duarte, M. \textit{Aristóteles e o Conhecimento}. Polis, 2019.},
    bibliografia complementar = {Silva, E. \textit{Os Pré-Socráticos}. Arqué, 2021.
    \newline Farias, V. \textit{A Origem da Filosofia}. Aurora, 2020.
    \newline Klein, F. \textit{Saber e Ser na Grécia Antiga}. Herma, 2018.},
    competências = {Compreender as origens da filosofia; identificar os principais temas e pensadores gregos; desenvolver leitura filosófica de textos clássicos.},
}

\PPCDefinaAtividade{rl-helen}{
    código = {1505002},
    nome = {Filosofia Helenística e Romana},
    objetivo = {Analisar as escolas filosóficas do período helenístico e romano, com ênfase na ética e na filosofia prática.},
    ementa = {Estoicismo, epicurismo e ceticismo. Ética, razão e natureza. A filosofia como modo de vida. Recepção romana do pensamento grego. Filosofia e consolação na Antiguidade Tardia.},
    pré-requisitos = {rl-intro},
    horas teóricas = 35,
    horas práticas = 15,
    horas extensionistas = 0,
    horas estágio = 0,
    departamento = DF,
    bibliografia básica = {Duarte, M. \textit{O Cuidado de Si}. Polis, 2020.
    \newline Mendes, C. \textit{Estoicos e Epicuristas}. Areté, 2019.
    \newline Silva, E. \textit{Sabedoria e Virtude}. Arqué, 2021.},
    bibliografia complementar = {Brito, A. \textit{O Homem Interior}. Herma, 2019.
    \newline Oliveira, R. \textit{Roma e o Pensamento Ético}. Thalassa, 2020.
    \newline Klein, F. \textit{Ceticismo e Liberdade}. Aurora, 2021.},
    competências = {Analisar as escolas helenísticas; compreender a filosofia como prática ética; interpretar textos filosóficos antigos.},
}

\PPCDefinaAtividade{rl-medieval}{
    código = {1505003},
    nome = {Filosofia Medieval e Patrística},
    objetivo = {Investigar o desenvolvimento do pensamento filosófico na Idade Média, a partir da síntese entre fé e razão.},
    ementa = {Filosofia cristã e herança clássica. Patrística e Escolástica. Agostinho e o problema do mal. Tomás de Aquino e a metafísica do ser. Relações entre teologia, lógica e cosmologia.},
    pré-requisitos = {rl-helen},
    horas teóricas = 40,
    horas práticas = 10,
    horas extensionistas = 0,
    horas estágio = 0,
    departamento = DF,
    bibliografia básica = {Oliveira, R. \textit{Fé e Razão na Idade Média}. Areté, 2020.
    \newline Mendes, C. \textit{Agostinho e o Tempo Interior}. Hélios, 2019.
    \newline Silva, E. \textit{Tomás de Aquino e o Ser}. Arqué, 2021.},
    bibliografia complementar = {Duarte, M. \textit{A Filosofia Cristã}. Polis, 2020.
    \newline Brito, A. \textit{O Pensamento Medieval}. Herma, 2018.
    \newline Farias, V. \textit{Luz, Verdade e Salvação}. Aurora, 2021.},
    competências = {Compreender a síntese fé-razão; analisar textos patrísticos e escolásticos; relacionar filosofia antiga e cristã.},
}

\PPCDefinaAtividade{rl-mod}{
    código = {1505004},
    nome = {Filosofia Moderna: Razão e Conhecimento},
    objetivo = {Estudar o pensamento moderno e as teorias do conhecimento, enfatizando o racionalismo e o empirismo.},
    ementa = {Renascimento e nova ciência. Racionalismo cartesiano e empirismo inglês. O problema da certeza e o ceticismo moderno. Kant e a síntese transcendental.},
    pré-requisitos = {rl-medieval},
    horas teóricas = 40,
    horas práticas = 10,
    horas extensionistas = 0,
    horas estágio = 0,
    departamento = DF,
    bibliografia básica = {Silva, E. \textit{Razão e Verdade}. Arqué, 2019.
    \newline Mendes, C. \textit{Da Dúvida à Certeza}. Hélios, 2020.
    \newline Duarte, M. \textit{Kant e a Crítica da Razão}. Polis, 2021.},
    bibliografia complementar = {Oliveira, R. \textit{Empirismo e Experiência}. Areté, 2018.
    \newline Brito, A. \textit{A Modernidade Filosófica}. Herma, 2021.
    \newline Farias, V. \textit{Racionalismo e Ciência}. Aurora, 2020.},
    competências = {Analisar as teorias modernas do conhecimento; compreender as rupturas epistemológicas; interpretar textos filosóficos modernos.},
}

\PPCDefinaAtividade{rl-contemp}{
    código = {1505005},
    nome = {Filosofia Contemporânea: Existência e Linguagem},
    objetivo = {Explorar as principais correntes do pensamento filosófico contemporâneo, com ênfase na fenomenologia, existencialismo e filosofia da linguagem.},
    ementa = {Fenomenologia e a questão da experiência. Existencialismo e liberdade. Linguagem, sentido e interpretação. Hermenêutica e desconstrução. Filosofia e ciência no século XX.},
    pré-requisitos = {rl-mod},
    horas teóricas = 40,
    horas práticas = 10,
    horas extensionistas = 0,
    horas estágio = 0,
    departamento = DF,
    bibliografia básica = {Mendes, C. \textit{O Ser e a Linguagem}. Hélios, 2020.
    \newline Oliveira, R. \textit{Fenomenologia e Mundo}. Areté, 2021.
    \newline Silva, E. \textit{Existência e Liberdade}. Arqué, 2019.},
    bibliografia complementar = {Duarte, M. \textit{Filosofia e Hermenêutica}. Polis, 2020.
    \newline Brito, A. \textit{Pensamento Contemporâneo}. Herma, 2021.
    \newline Farias, V. \textit{O Fim da Metafísica?}. Aurora, 2020.},
    competências = {Compreender as correntes filosóficas contemporâneas; analisar conceitos de existência, linguagem e sentido; aplicar leitura hermenêutica de textos filosóficos.},
}

\PPCDefinaAtividade{rl-etica}{
    código = {1505006},
    nome = {Ética e Filosofia Moral},
    objetivo = {Refletir sobre os fundamentos da ação ética e suas transformações ao longo da história do pensamento filosófico.},
    ementa = {Virtude e felicidade na Antiguidade. Moral cristã e dever moderno. Éticas consequencialistas, deontológicas e das virtudes. Problemas éticos contemporâneos. Ética aplicada e bioética.},
    pré-requisitos = {rl-contemp},
    horas teóricas = 35,
    horas práticas = 15,
    horas extensionistas = 10,
    horas estágio = 0,
    departamento = DF,
    bibliografia básica = {Silva, E. \textit{Ética e Virtude}. Arqué, 2019.
    \newline Duarte, M. \textit{O Dever e o Bem}. Polis, 2020.
    \newline Mendes, C. \textit{Filosofia Moral Contemporânea}. Hélios, 2021.},
    bibliografia complementar = {Brito, A. \textit{A Ética no Mundo Atual}. Herma, 2018.
    \newline Farias, V. \textit{Bioética e Responsabilidade}. Aurora, 2020.
    \newline Oliveira, R. \textit{A Ação Justa}. Areté, 2021.},
    competências = {Analisar teorias éticas; compreender dilemas morais contemporâneos; aplicar princípios filosóficos à reflexão ética.},
}

\PPCDefinaAtividade{rl-estetica}{
    código = {1505007},
    nome = {Estética e Filosofia da Arte},
    objetivo = {Investigar as concepções filosóficas do belo, da arte e da criação, da Antiguidade à contemporaneidade.},
    ementa = {Origem da estética filosófica. O belo em Platão e Aristóteles. Arte e subjetividade no idealismo. Crítica estética moderna. Filosofia da arte contemporânea e estética da recepção.},
    pré-requisitos = {rl-contemp},
    horas teóricas = 35,
    horas práticas = 15,
    horas extensionistas = 10,
    horas estágio = 0,
    departamento = DF,
    bibliografia básica = {Mendes, C. \textit{O Belo e o Pensamento}. Hélios, 2020.
    \newline Oliveira, R. \textit{Arte e Filosofia}. Areté, 2018.
    \newline Duarte, M. \textit{Estética e Criação}. Polis, 2021.},
    bibliografia complementar = {Silva, E. \textit{O Sentido do Belo}. Arqué, 2019.
    \newline Brito, A. \textit{Estética Contemporânea}. Herma, 2021.
    \newline Farias, V. \textit{Arte e Percepção}. Aurora, 2020.},
    competências = {Compreender os fundamentos filosóficos da arte; analisar concepções do belo; relacionar arte, estética e pensamento crítico.},
}

\PPCDefinaAtividade{rl-logica}{
    código = {1505008},
    nome = {Lógica e Argumentação Filosófica},
    objetivo = {Desenvolver o raciocínio lógico e a análise de argumentos, com base em princípios formais e filosóficos.},
    ementa = {Proposições, inferência e validade. Silogismo e lógica proposicional. Argumentação e falácias. Lógica simbólica básica. Aplicações da lógica à análise filosófica e científica.},
    pré-requisitos = {},
    horas teóricas = 40,
    horas práticas = 10,
    horas extensionistas = 0,
    horas estágio = 0,
    departamento = DF,
    bibliografia básica = {Silva, E. \textit{Lógica e Linguagem}. Arqué, 2018.
    \newline Duarte, M. \textit{Introdução à Lógica}. Polis, 2019.
    \newline Oliveira, R. \textit{Argumentação e Racionalidade}. Areté, 2020.},
    bibliografia complementar = {Mendes, C. \textit{Pensar com Clareza}. Hélios, 2021.
    \newline Brito, A. \textit{O Raciocínio Lógico}. Herma, 2019.
    \newline Farias, V. \textit{Lógica e Filosofia}. Aurora, 2020.},
    competências = {Reconhecer formas válidas de raciocínio; aplicar lógica à argumentação filosófica; avaliar a consistência de discursos e textos.},
}
