\PPCDefinaAtividade{fil-pre}{
    código = {1202001},
    nome = {Introdução à Filosofia e ao Pensamento Pré-Socrático},
    nome abreviado = {Intr. à Filosofia e ao Pensam. Pré-Socrático},
    objetivo = {Apresentar as origens da filosofia ocidental, analisando os primeiros pensadores gregos e suas concepções sobre natureza, ser e conhecimento.},
    ementa = {Estudo das escolas jônica, eleata e pitagórica. O surgimento do pensamento racional na Grécia. Relações entre mito e logos. Leitura de fragmentos e comentários de autores pré-socráticos.},
    pré-requisitos = {},
    horas teóricas = 40,
    horas práticas = 20,
    horas extensionistas = 0,
    horas estágio = 0,
    departamento = DF,
    bibliografia básica = {Mendes, C. \textit{Os Primeiros Filósofos do Ocidente}. Arqué, 2018.
    \newline Oliveira, R. \textit{Do Mito ao Logos}. Hélios, 2017.
    \newline Duarte, M. \textit{Heráclito e o Fogo do Pensar}. Polis, 2020.
    \newline Lemos, P. \textit{Parmênides e o Ser}. Areté, 2019.},
    bibliografia complementar = {Santos, P. \textit{A Natureza e o Pensamento Pré-Socrático}. Aurora, 2016.
    \newline Brito, A. \textit{Fragmentos e Cosmologia Antiga}. Herma, 2018.
    \newline Farias, V. \textit{Entre Mito e Razão}. Thalassa, 2021.},
    competências = {Compreender a transição do pensamento mítico ao racional; identificar os principais conceitos dos filósofos pré-socráticos; desenvolver leitura crítica de fragmentos antigos.},
}

\PPCDefinaAtividade{fil-platao}{
    código = {1202002},
    nome = {Platão e o Mundo das Ideias},
    objetivo = {Investigar o pensamento platônico e sua influência sobre a tradição filosófica ocidental, com ênfase na teoria das ideias e na ética do conhecimento.},
    ementa = {Estudo dos diálogos platônicos: epistemologia, política e estética. A teoria das ideias, o mito da caverna e o conceito de alma. A relação entre razão e transcendência. Leitura comentada de trechos selecionados.},
    pré-requisitos = {fil-pre},
    horas teóricas = 35,
    horas práticas = 25,
    horas extensionistas = 0,
    horas estágio = 0,
    departamento = DF,
    bibliografia básica = {Duarte, M. \textit{A Filosofia de Platão}. Areté, 2019.
    \newline Oliveira, R. \textit{Platão e o Horizonte das Ideias}. Polis, 2016.
    \newline Mendes, C. \textit{Diálogos e Formas do Saber}. Thalassa, 2020.
    \newline Lemos, P. \textit{O Mito e o Mundo Platônico}. Hélios, 2018.},
    bibliografia complementar = {Santos, P. \textit{A Ética da Alma em Platão}. Herma, 2017.
    \newline Brito, A. \textit{O Sócrates dos Diálogos}. Aurora, 2019.
    \newline Klein, F. \textit{Razão e Beleza na Filosofia Antiga}. Arqué, 2021.},
    competências = {Analisar os conceitos fundamentais do pensamento platônico; compreender a relação entre ética, estética e política; interpretar textos filosóficos de modo argumentativo.},
}

\PPCDefinaAtividade{fil-aristo}{
    código = {1202003},
    nome = {Aristóteles: Lógica, Ética e Física},
    objetivo = {Estudar o sistema filosófico aristotélico e sua articulação entre lógica, natureza e moral.},
    ementa = {Análise da \textit{Organon}, da \textit{Ética a Nicômaco} e da \textit{Física}. O conceito de substância, causa e finalidade. A noção de virtude e o papel da razão prática. A herança aristotélica no pensamento científico.},
    pré-requisitos = {fil-platao},
    horas teóricas = 40,
    horas práticas = 20,
    horas extensionistas = 0,
    horas estágio = 0,
    departamento = DF,
    bibliografia básica = {Mendes, C. \textit{Aristóteles e o Movimento do Pensar}. Polis, 2018.
    \newline Duarte, M. \textit{A Lógica e o Ser em Aristóteles}. Areté, 2017.
    \newline Oliveira, R. \textit{A Física do Mundo Antigo}. Hélios, 2019.
    \newline Lemos, P. \textit{Virtude e Medida}. Arqué, 2020.},
    bibliografia complementar = {Santos, P. \textit{A Ética e o Bem em Aristóteles}. Aurora, 2016.
    \newline Brito, A. \textit{Comentário à Física Aristotélica}. Herma, 2019.
    \newline Klein, F. \textit{O Pensamento Analítico Antigo}. Thalassa, 2021.},
    competências = {Dominar noções fundamentais da filosofia aristotélica; relacionar lógica e ética à noção de natureza; aplicar o pensamento clássico à análise racional de problemas.},
}

\PPCDefinaAtividade{fil-sagrado}{
    código = {1202004},
    nome = {O Sagrado e o Natural no Pensamento Antigo},
    nome = {O Sagrado e o Natural no Pensam. Antigo},
    objetivo = {Explorar as concepções de natureza, divino e cosmos na filosofia e religião antigas, enfatizando a continuidade entre o natural e o sagrado.},
    ementa = {Estudo da cosmologia filosófica e religiosa grega. As noções de ordem, harmonia e sacralidade. Relações entre filosofia, rito e ciência. Reflexão sobre o pensamento simbólico antigo e suas permanências.},
    pré-requisitos = {fil-aristo},
    horas teóricas = 30,
    horas práticas = 20,
    horas extensionistas = 10,
    horas estágio = 0,
    departamento = DF,
    bibliografia básica = {Oliveira, R. \textit{Cosmos e Divindade na Antiguidade}. Polis, 2017.
    \newline Mendes, C. \textit{O Pensamento Religioso Grego}. Areté, 2018.
    \newline Duarte, M. \textit{Filosofia e Mistério}. Hélios, 2020.
    \newline Lemos, P. \textit{O Natural e o Divino}. Arqué, 2021.},
    bibliografia complementar = {Brito, A. \textit{Rito e Razão no Mundo Antigo}. Herma, 2019.
    \newline Farias, V. \textit{O Sagrado como Linguagem}. Aurora, 2018.
    \newline Klein, F. \textit{Cosmologia Simbólica Antiga}. Thalassa, 2020.},
    competências = {Reconhecer o vínculo entre natureza e transcendência; analisar representações do sagrado; interpretar o papel da filosofia na compreensão do cosmos antigo.},
}

\PPCDefinaAtividade{fil-neoplat}{
    código = {1202005},
    nome = {Neoplatonismo e a Unidade do Ser},
    objetivo = {Compreender o desenvolvimento do pensamento neoplatônico e sua síntese metafísica, destacando a influência de Plotino e seus seguidores.},
    ementa = {Estudo das doutrinas neoplatônicas: o Uno, o Intelecto e a Alma. A relação entre ser e emanação. Dimensões místicas e racionais do pensamento tardio antigo. Leituras de Plotino e Porfírio.},
    pré-requisitos = {fil-platao},
    horas teóricas = 35,
    horas práticas = 25,
    horas extensionistas = 0,
    horas estágio = 0,
    departamento = DF,
    bibliografia básica = {Duarte, M. \textit{Plotino e a Unidade do Ser}. Arqué, 2019.
    \newline Mendes, C. \textit{A Filosofia da Emanação}. Hélios, 2020.
    \newline Oliveira, R. \textit{Do Uno à Mente}. Polis, 2017.
    \newline Lemos, P. \textit{A Tradição Neoplatônica}. Areté, 2018.},
    bibliografia complementar = {Santos, P. \textit{O Espírito e a Matéria no Neoplatonismo}. Herma, 2016.
    \newline Brito, A. \textit{Plotino e a Filosofia Tardia}. Aurora, 2019.
    \newline Klein, F. \textit{Metafísica da Unidade}. Thalassa, 2021.},
    competências = {Interpretar textos neoplatônicos; compreender a concepção de unidade e transcendência; reconhecer a influência do neoplatonismo na filosofia posterior.},
}

\PPCDefinaAtividade{fil-helen}{
    código = {1202006},
    nome = {Filosofia Helenística e Sabedoria Prática},
    objetivo = {Estudar as escolas filosóficas helenísticas e sua ênfase na ética como forma de vida, abordando estoicismo, epicurismo e ceticismo.},
    ementa = {Análise comparada das principais escolas do período helenístico. A busca pela ataraxia, pela razão universal e pela moderação dos desejos. Atualidade das filosofias práticas antigas.},
    pré-requisitos = {fil-aristo},
    horas teóricas = 30,
    horas práticas = 20,
    horas extensionistas = 10,
    horas estágio = 0,
    departamento = DF,
    bibliografia básica = {Mendes, C. \textit{A Filosofia da Serenidade}. Polis, 2020.
    \newline Duarte, M. \textit{Estoicos, Epicuristas e Céticos}. Areté, 2018.
    \newline Oliveira, R. \textit{Viver Bem na Antiguidade}. Hélios, 2019.
    \newline Lemos, P. \textit{Sabedoria Prática e Ética Antiga}. Arqué, 2016.},
    bibliografia complementar = {Santos, P. \textit{A Vida Filosófica no Mundo Helenístico}. Herma, 2017.
    \newline Brito, A. \textit{Filosofia como Terapia}. Aurora, 2020.
    \newline Farias, V. \textit{A Ética do Equilíbrio}. Thalassa, 2021.},
    competências = {Compreender as principais escolas helenísticas; analisar a relação entre filosofia e modo de vida; aplicar princípios éticos antigos à reflexão contemporânea.},
}

\PPCDefinaAtividade{fil-antimod}{
    código = {1202007},
    nome = {Filosofia Antiga e Crítica da Modernidade},
    objetivo = {Examinar como conceitos da filosofia antiga são retomados para questionar fundamentos da modernidade e do pensamento técnico.},
    ementa = {A presença do pensamento grego em filósofos modernos e contemporâneos. Crítica da racionalidade instrumental a partir da tradição clássica. Leituras de Nietzsche, Heidegger e Foucault em diálogo com os antigos.},
    pré-requisitos = {fil-pre},
    horas teóricas = 35,
    horas práticas = 15,
    horas extensionistas = 10,
    horas estágio = 0,
    departamento = DF,
    bibliografia básica = {Oliveira, R. \textit{Os Antigos e os Modernos}. Polis, 2018.
    \newline Mendes, C. \textit{Heidegger e o Pensamento Grego}. Hélios, 2020.
    \newline Duarte, M. \textit{Nietzsche e a Antiguidade}. Arqué, 2019.
    \newline Lemos, P. \textit{A Crítica da Modernidade Clássica}. Areté, 2021.},
    bibliografia complementar = {Santos, P. \textit{A Filosofia Antiga e o Mundo Contemporâneo}. Aurora, 2017.
    \newline Brito, A. \textit{Tradição e Ruptura no Pensar}. Herma, 2020.
    \newline Klein, F. \textit{Entre o Ser Antigo e o Ser Moderno}. Thalassa, 2018.},
    competências = {Analisar a presença da filosofia antiga na crítica da modernidade; compreender o diálogo entre tradição e inovação; desenvolver pensamento comparativo entre épocas.},
}
