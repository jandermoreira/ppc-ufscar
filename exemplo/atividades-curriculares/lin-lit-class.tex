\PPCDefinaAtividade{lin-grego}{
    código = {1101001},
    nome = {Introdução ao Grego Antigo},
    objetivo = {Introduzir os fundamentos da língua grega clássica e desenvolver a leitura e tradução de textos curtos, relacionando linguagem e cultura.},
    ementa = {Estudo do alfabeto grego, fonética e morfologia básica. Elementos essenciais da sintaxe. Leitura e tradução de trechos de autores clássicos. Relações entre língua, pensamento e cultura na Grécia Antiga.},
    pré-requisitos = {},
    horas teóricas = 40,
    horas práticas = 20,
    horas extensionistas = 0,
    horas estágio = 0,
    departamento = DL,
    bibliografia básica = {Heliodoros, A. \textit{Gramática Elementar do Grego Clássico}. Polis Editora, 2015.
    \newline Mendes, C. \textit{O Verbo Grego e o Pensamento}. Arqué, 2018.
    \newline Lemos, P. \textit{Introdução à Leitura de Textos Gregos}. Hélios, 2020.
    \newline Oliveira, R. \textit{Língua e Cosmos na Grécia Antiga}. Thalassa, 2016.},
    bibliografia complementar = {Duarte, M. \textit{Filologia e Cultura Grega}. Herma, 2014.
    \newline Klein, F. \textit{O Som e o Sentido no Grego Clássico}. Areté, 2019.
    \newline Sanches, J. \textit{Gramática Histórica do Grego}. Ed. Mythos, 2017.},
    competências = {Leitura e compreensão de textos gregos básicos; domínio inicial da morfologia e sintaxe; reconhecimento de elementos culturais na linguagem.},
}

\PPCDefinaAtividade{lin-latim}{
    código = {1101002},
    nome = {Introdução ao Latim},
    objetivo = {Proporcionar o domínio introdutório da língua latina e sua importância para as línguas e culturas ocidentais.},
    ementa = {Morfologia e sintaxe elementar da língua latina. Vocabulário essencial. Leitura de textos curtos de autores clássicos. O latim como língua de saber e transmissão cultural.},
    pré-requisitos = {},
    horas teóricas = 35,
    horas práticas = 25,
    horas extensionistas = 0,
    horas estágio = 0,
    departamento = DL,
    bibliografia básica = {Martinus, L. \textit{Latinitas Viva}. Verbum, 2016.
    \newline Santos, P. \textit{Introdução à Gramática Latina}. Hélios, 2018.
    \newline Oliveira, R. \textit{O Latim e a Formação do Pensamento Ocidental}. Polis, 2020.
    \newline Mendes, C. \textit{Textos Latinos Comentados}. Thalassa, 2019.},
    bibliografia complementar = {Brito, A. \textit{Gramática Comparada do Grego e do Latim}. Herma, 2017.
    \newline Dantas, F. \textit{O Mundo Romano e sua Linguagem}. Ed. Aurora, 2021.
    \newline Aramis, G. \textit{Estudos Clássicos e Filológicos}. Arqué, 2014.},
    competências = {Compreender estruturas básicas do latim; aplicar princípios de tradução elementar; reconhecer a herança latina nas línguas modernas.},
}

\PPCDefinaAtividade{lin-trad}{
    código = {1101003},
    nome = {Oficina de Tradução e Interpretação Textual},
    nome abreviado = {Oficina de Trad. e Interpr. Textual},
    objetivo = {Desenvolver habilidades de tradução e análise textual em línguas clássicas, articulando fidelidade e recriação interpretativa.},
    ementa = {Fundamentos teóricos da tradução. Exercícios práticos de tradução de trechos gregos e latinos. Estilo, sentido e adaptação. Leitura comparada de diferentes versões e comentários.},
    pré-requisitos = {lin-grego, lin-latim},
    horas teóricas = 20,
    horas práticas = 30,
    horas extensionistas = 10,
    horas estágio = 0,
    departamento = DL,
    bibliografia básica = {Silva, D. \textit{Teoria da Tradução Clássica}. Arqué, 2018.
    \newline Lemos, P. \textit{Oficina de Tradução Grego-Latina}. Polis, 2020.
    \newline Duarte, M. \textit{Entre Línguas: Tradução e Cultura}. Hélios, 2017.
    \newline Mendes, C. \textit{Interpretação e Sentido nos Textos Antigos}. Areté, 2019.},
    bibliografia complementar = {Ramos, T. \textit{A Arte de Traduzir os Clássicos}. Aurora, 2020.
    \newline Castro, L. \textit{Traduzir o Mundo Antigo}. Thalassa, 2018.
    \newline Oliveira, R. \textit{Linguagem e Transmissão do Saber}. Polis, 2015.},
    competências = {Traduzir textos curtos do grego e latim; aplicar princípios de crítica textual; refletir sobre tradução como recriação cultural.},
}

\PPCDefinaAtividade{lit-grega}{
    código = {1101004},
    nome = {Literatura Grega: Épica e Tragédia},
    objetivo = {Compreender as formas literárias da épica e da tragédia gregas como expressões do pensamento e da cultura clássica.},
    ementa = {Leitura e análise de trechos da \textit{Ilíada}, \textit{Odisseia} e tragédias de Ésquilo, Sófocles e Eurípides. Estrutura narrativa, mito e valor ético. A literatura como forma de pensamento coletivo.},
    pré-requisitos = {lin-grego},
    horas teóricas = 40,
    horas práticas = 20,
    horas extensionistas = 0,
    horas estágio = 0,
    departamento = DL,
    bibliografia básica = {Mendes, C. \textit{A Tragédia Grega e o Destino Humano}. Hélios, 2019.
    \newline Oliveira, R. \textit{Leituras de Homero}. Arqué, 2016.
    \newline Duarte, M. \textit{Poética e Filosofia na Grécia Antiga}. Thalassa, 2018.
    \newline Klein, F. \textit{Épica e Sabedoria Grega}. Areté, 2020.},
    bibliografia complementar = {Lemos, P. \textit{Estruturas do Drama Grego}. Herma, 2017.
    \newline Santos, P. \textit{Mitologia e Teatro na Antiguidade}. Polis, 2015.
    \newline Nogueira, E. \textit{A Voz dos Heróis}. Aurora, 2021.},
    competências = {Interpretar textos literários gregos; relacionar mitologia, linguagem e ética; analisar o valor simbólico da literatura antiga.},
}

\PPCDefinaAtividade{lin-ingant}{
    código = {1101005},
    nome = {Inglês Antigo e Cultura Anglo-Saxônica},
    objetivo = {Introduzir o estudo do inglês antigo e do universo cultural e simbólico da tradição anglo-saxônica.},
    ementa = {Aspectos linguísticos e históricos do inglês antigo. Leitura de trechos de \textit{Beowulf} e crônicas medievais. Relações entre heranças clássicas e nórdicas. Identidade, mito e herói.},
    pré-requisitos = {},
    horas teóricas = 30,
    horas práticas = 20,
    horas extensionistas = 10,
    horas estágio = 0,
    departamento = DL,
    bibliografia básica = {Smith, A. \textit{Old English Grammar and Texts}. Northwind Press, 2018.
    \newline Duarte, M. \textit{Mitologia Germânica e Cultura Medieval}. Arqué, 2020.
    \newline Oliveira, R. \textit{Raízes Clássicas do Mundo Nórdico}. Hélios, 2021.
    \newline Mendes, C. \textit{A Palavra e o Herói Anglo-Saxão}. Thalassa, 2017.},
    bibliografia complementar = {Santos, P. \textit{Beowulf e a Tradição Épica Europeia}. Areté, 2016.
    \newline Lemos, P. \textit{Do Latim ao Inglês Antigo}. Polis, 2018.
    \newline Farias, V. \textit{Cultura e Imaginário Nórdico}. Aurora, 2020.},
    competências = {Reconhecer elementos linguísticos do inglês antigo; compreender a cultura anglo-saxônica e suas conexões com o mundo clássico.},
}

\PPCDefinaAtividade{lin-esttex}{
    código = {1101006},
    nome = {Estudo Dirigido de Textos Clássicos},
    objetivo = {Aprofundar a leitura e a interpretação de textos gregos e latinos, com ênfase em análise filológica e literária.},
    ementa = {Leitura orientada de textos selecionados de Homero, Virgílio, Platão, Cícero e Horácio. Comentário filológico e temático. Exercícios de tradução e análise comparada.},
    pré-requisitos = {lin-trad},
    horas teóricas = 20,
    horas práticas = 30,
    horas extensionistas = 10,
    horas estágio = 0,
    departamento = DL,
    bibliografia básica = {Mendes, C. \textit{Clássicos em Leitura}. Arqué, 2019.
    \newline Duarte, M. \textit{Textos Gregos e Latinos Comentados}. Polis, 2020.
    \newline Oliveira, R. \textit{Estudos de Filologia Clássica}. Hélios, 2018.
    \newline Lemos, P. \textit{A Arte de Ler os Antigos}. Thalassa, 2017.},
    bibliografia complementar = {Sanches, J. \textit{Métodos de Leitura Filológica}. Herma, 2019.
    \newline Santos, P. \textit{Interpretação e Tradução dos Clássicos}. Areté, 2021.
    \newline Ramos, T. \textit{Linguagem, Ética e Estilo}. Aurora, 2018.},
    competências = {Ler e comentar textos clássicos em língua original; aplicar métodos filológicos; integrar análise textual e cultural.},
}

\PPCDefinaAtividade{lit-latina}{
    código = {1101007},
    nome = {Literatura Latina: Poesia e Retórica},
    objetivo = {Explorar a literatura latina como expressão poética e oratória, destacando suas formas, temas e influências.},
    ementa = {Leitura e análise de textos de Virgílio, Horácio, Ovídio e Cícero. A estrutura poética e o discurso retórico. Relações entre forma, pensamento e cultura.},
    pré-requisitos = {lin-latim},
    horas teóricas = 40,
    horas práticas = 20,
    horas extensionistas = 0,
    horas estágio = 0,
    departamento = DL,
    bibliografia básica = {Oliveira, R. \textit{Poesia e Poder em Roma}. Polis, 2017.
    \newline Mendes, C. \textit{A Retórica de Cícero}. Areté, 2018.
    \newline Duarte, M. \textit{Lírica Latina e Filosofia}. Hélios, 2020.
    \newline Lemos, P. \textit{Textos Poéticos da Roma Antiga}. Arqué, 2016.},
    bibliografia complementar = {Santos, P. \textit{Ovídio e a Imaginação Mítica}. Aurora, 2019.
    \newline Klein, F. \textit{Estilo e Persuasão na Antiguidade}. Herma, 2018.
    \newline Brito, A. \textit{A Palavra e o Império}. Thalassa, 2021.},
    competências = {Interpretar textos poéticos e oratórios; compreender o papel da linguagem na cultura romana; aplicar conceitos de retórica clássica.},
}

\PPCDefinaAtividade{lin-poetrad}{
    código = {1101008},
    nome = {Poéticas da Tradução: Do Mundo Antigo à Modernidade},
    nome = {Poéticas da Tradução},
    objetivo = {Examinar a tradução como prática estética e filosófica, enfatizando a recriação dos textos clássicos em contextos modernos.},
    ementa = {Estudo de teorias da tradução. Comparação entre versões antigas e contemporâneas de textos gregos e latinos. Reflexão sobre anacronismo, estilo e recriação poética.},
    pré-requisitos = {lin-trad},
    horas teóricas = 25,
    horas práticas = 25,
    horas extensionistas = 10,
    horas estágio = 0,
    departamento = DL,
    bibliografia básica = {Silva, D. \textit{Tradução como Poética}. Arqué, 2018.
    \newline Mendes, C. \textit{A Arte de Recriar os Clássicos}. Hélios, 2020.
    \newline Duarte, M. \textit{Tradução e Filosofia Antiga}. Polis, 2017.
    \newline Oliveira, R. \textit{Traduzir e Pensar}. Areté, 2019.},
    bibliografia complementar = {Lemos, P. \textit{As Palavras que Retornam}. Thalassa, 2021.
    \newline Castro, L. \textit{O Tempo da Tradução}. Aurora, 2018.
    \newline Ramos, T. \textit{Estilo, Memória e Tradução}. Herma, 2016.},
    competências = {Analisar criticamente traduções clássicas e modernas; compreender a tradução como recriação estética; propor reflexões intertemporais sobre o texto.},
}
