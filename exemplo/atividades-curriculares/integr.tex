\PPCDefinaAtividade{ai-leitura}{
    código = {1707001},
    nome = {Leitura e Escrita Acadêmica},
    objetivo = {Desenvolver habilidades de leitura crítica, produção textual e argumentação em linguagem acadêmica, integrando reflexão e expressão escrita.},
    ementa = {Estrutura do texto acadêmico. Coesão, coerência e estilo científico. Estratégias de leitura crítica e síntese. Práticas de escrita e revisão de trabalhos acadêmicos. Ética e autoria.},
    pré-requisitos = {},
    horas teóricas = 30,
    horas práticas = 20,
    horas extensionistas = 10,
    horas estágio = 0,
    departamento = DL,
    bibliografia básica = {Nogueira, L. \textit{Escrever com Clareza}. Ânfora, 2019.
    \newline Lima, P. \textit{Textos e Argumentos}. Arcádia, 2020.
    \newline Rodrigues, T. \textit{Leitura Crítica e Escrita Acadêmica}. Lúmen, 2021.
    \newline Melo, R. \textit{O Texto Científico}. Thalassa, 2018.},
    bibliografia complementar = {Castro, A. \textit{Redação e Pesquisa}. Solaris, 2021.
    \newline Dias, M. \textit{Comunicação e Conhecimento}. Arqué, 2019.
    \newline Brito, A. \textit{Estilo e Clareza no Texto Acadêmico}. Herma, 2020.},
    competências = {Ler criticamente textos acadêmicos; estruturar e redigir textos científicos; comunicar ideias com precisão e coerência.},
}

\PPCDefinaAtividade{ai-labcultura}{
    código = {1707002},
    nome = {Laboratório de Cultura Material Antiga},
    objetivo = {Proporcionar contato prático com objetos e vestígios do mundo antigo, desenvolvendo a análise cultural e histórica do material arqueológico.},
    ementa = {Introdução à cultura material. Objetos, inscrições e iconografia. Métodos de análise e conservação. Representações simbólicas e práticas sociais na Antiguidade. Projetos de registro e difusão cultural.},
    pré-requisitos = {},
    horas teóricas = 25,
    horas práticas = 25,
    horas extensionistas = 10,
    horas estágio = 0,
    departamento = DH,
    bibliografia básica = {Melo, R. \textit{Objetos e Memória}. Thalassa, 2020.
    \newline Silva, J. \textit{Cultura Material e Sociedade Antiga}. Polis, 2019.
    \newline Castro, A. \textit{Arqueologia e Interpretação}. Solaris, 2021.
    \newline Lima, P. \textit{Vestígios do Passado}. Arcádia, 2018.},
    bibliografia complementar = {Dias, M. \textit{Cultura e Imagem}. Arqué, 2021.
    \newline Rodrigues, T. \textit{A História dos Objetos}. Lúmen, 2020.
    \newline Nogueira, L. \textit{Memória Arqueológica}. Ânfora, 2021.},
    competências = {Analisar objetos do mundo antigo; compreender a relação entre materialidade e cultura; aplicar métodos de registro e difusão patrimonial.},
}

\PPCDefinaAtividade{ai-proj1}{
    código = {1707003},
    nome = {Projeto Interdisciplinar I (Pesquisa e Extensão)},
    nome = {Projeto Interdisciplinar I},
    objetivo = {Integrar conhecimentos de diferentes áreas do curso por meio de projetos de pesquisa e extensão voltados à difusão do saber clássico.},
    ementa = {Concepção e elaboração de projetos interdisciplinares. Pesquisa bibliográfica, metodologia e cronograma. Atividades extensionistas e divulgação científica. Produção de relatórios e apresentações.},
    pré-requisitos = {},
    horas teóricas = 20,
    horas práticas = 20,
    horas extensionistas = 20,
    horas estágio = 0,
    departamento = DH,
    bibliografia básica = {Rodrigues, T. \textit{Metodologia Interdisciplinar}. Lúmen, 2020.
    \newline Lima, P. \textit{Pesquisa em Humanidades}. Arcádia, 2019.
    \newline Melo, R. \textit{Extensão e Cultura}. Thalassa, 2021.
    \newline Castro, A. \textit{Projetos Culturais e Educação}. Solaris, 2018.},
    bibliografia complementar = {Dias, M. \textit{Pesquisa Aplicada}. Arqué, 2020.
    \newline Nogueira, L. \textit{Cultura e Interação}. Ânfora, 2021.
    \newline Brito, A. \textit{Aprendizagem por Projetos}. Herma, 2020.},
    competências = {Planejar e executar projetos interdisciplinares; aplicar metodologia científica; comunicar resultados de pesquisa e extensão.},
}

\PPCDefinaAtividade{ai-proj2}{
    código = {1707004},
    nome = {Projeto Interdisciplinar II (Educação, Cultura e Tradução)},
    nome abreviado = {Projeto Interdisciplinar II},
    objetivo = {Aprofundar a integração entre ensino, pesquisa e extensão por meio de projetos aplicados à tradução e difusão cultural dos saberes antigos.},
    ementa = {Desenvolvimento de projetos de tradução, reinterpretação ou mediação cultural. Elaboração de materiais didáticos e ações formativas. Práticas de extensão com comunidades e instituições culturais.},
    pré-requisitos = {ai-proj1},
    horas teóricas = 20,
    horas práticas = 20,
    horas extensionistas = 20,
    horas estágio = 0,
    departamento = DH,
    bibliografia básica = {Rodrigues, T. \textit{Tradução e Cultura}. Lúmen, 2021.
    \newline Lima, P. \textit{Educação e Saberes Antigos}. Arcádia, 2019.
    \newline Melo, R. \textit{Práticas Extensionistas em Humanidades}. Thalassa, 2020.
    \newline Castro, A. \textit{Cultura e Ensino Clássico}. Solaris, 2021.},
    bibliografia complementar = {Dias, M. \textit{Ação Cultural e Sociedade}. Arqué, 2019.
    \newline Nogueira, L. \textit{Ensinar o Clássico Hoje}. Ânfora, 2020.
    \newline Brito, A. \textit{Educação e Tradição}. Herma, 2021.},
    competências = {Integrar pesquisa e extensão; promover ações educativas e culturais; aplicar conhecimentos clássicos a contextos sociais contemporâneos.},
}

\PPCDefinaAtividade{ai-metodologia}{
    código = {1707005},
    nome = {Metodologia da Pesquisa em Humanidades},
    objetivo = {Fornecer fundamentos teóricos e práticos para o desenvolvimento de pesquisas científicas em Humanidades, com foco na área de estudos clássicos.},
    ementa = {Natureza e objetivos da pesquisa em Humanidades. Métodos qualitativos e hermenêuticos. Construção de problemas e hipóteses. Revisão bibliográfica, análise e redação científica. Ética na pesquisa.},
    pré-requisitos = {},
    horas teóricas = 40,
    horas práticas = 10,
    horas extensionistas = 10,
    horas estágio = 0,
    departamento = DF,
    bibliografia básica = {Lima, P. \textit{Métodos em Humanidades}. Arcádia, 2021.
    \newline Rodrigues, T. \textit{Pesquisa e Interpretação}. Lúmen, 2020.
    \newline Melo, R. \textit{Caminhos da Investigação}. Thalassa, 2019.
    \newline Dias, M. \textit{Ética e Método}. Arqué, 2018.},
    bibliografia complementar = {Castro, A. \textit{Introdução à Pesquisa Filosófica}. Solaris, 2020.
    \newline Nogueira, L. \textit{O Texto Científico nas Humanidades}. Ânfora, 2021.
    \newline Brito, A. \textit{Pesquisa e Escrita}. Herma, 2020.},
    competências = {Compreender metodologias de pesquisa em Humanidades; elaborar projetos de investigação; redigir textos científicos de forma crítica e ética.},
}

\PPCDefinaAtividade{ai-seminario}{
    código = {1707006},
    nome = {Seminário Temático Avançado},
    objetivo = {Aprofundar o estudo de temas específicos da Antiguidade ou de sua recepção moderna, integrando as diversas áreas temáticas do curso.},
    ementa = {Discussão de temas interdisciplinares avançados em grupos de pesquisa. Leitura de textos fundamentais. Produção de artigos e ensaios temáticos. Apresentações públicas e rodas de debate.},
    pré-requisitos = {},
    horas teóricas = 25,
    horas práticas = 25,
    horas extensionistas = 10,
    horas estágio = 0,
    departamento = DH,
    bibliografia básica = {Melo, R. \textit{Estudos Avançados em Humanidades}. Thalassa, 2020.
    \newline Lima, P. \textit{Tópicos em Cultura Clássica}. Arcádia, 2021.
    \newline Rodrigues, T. \textit{Temas Contemporâneos da Antiguidade}. Lúmen, 2019.
    \newline Castro, A. \textit{Leituras e Debates}. Solaris, 2018.},
    bibliografia complementar = {Dias, M. \textit{Humanidades em Perspectiva}. Arqué, 2020.
    \newline Nogueira, L. \textit{Pesquisa e Interpretação}. Ânfora, 2021.
    \newline Brito, A. \textit{Ensaios e Crítica Cultural}. Herma, 2021.},
    competências = {Aprofundar temas específicos das Humanidades; integrar pesquisa, escrita e exposição oral; participar de debates e comunicações científicas.},
}

\PPCDefinaAtividade{ai-producao}{
    código = {1707007},
    nome = {Oficinas de Produção Científica e Cultural},
    objetivo = {Estimular a criação de produtos científicos, culturais e educativos derivados das pesquisas desenvolvidas no curso.},
    ementa = {Planejamento e produção de artigos, ensaios, traduções, podcasts, exposições e materiais didáticos. Mediação cultural e divulgação científica. Avaliação de impacto e comunicação pública do conhecimento.},
    pré-requisitos = {ai-seminario},
    horas teóricas = 20,
    horas práticas = 30,
    horas extensionistas = 10,
    horas estágio = 0,
    departamento = DH,
    bibliografia básica = {Lima, P. \textit{Produção de Conhecimento e Cultura}. Arcádia, 2021.
    \newline Rodrigues, T. \textit{Divulgação Científica em Humanidades}. Lúmen, 2020.
    \newline Melo, R. \textit{Oficinas de Criação Cultural}. Thalassa, 2019.
    \newline Dias, M. \textit{Pesquisa e Comunicação Pública}. Arqué, 2020.},
    bibliografia complementar = {Castro, A. \textit{Ciência e Cultura}. Solaris, 2021.
    \newline Nogueira, L. \textit{Narrar o Saber}. Ânfora, 2018.
    \newline Brito, A. \textit{Criar e Comunicar}. Herma, 2020.},
    competências = {Transformar pesquisa em produtos culturais; comunicar resultados científicos; desenvolver práticas de mediação e divulgação.},
}

\PPCDefinaAtividade{ai-tcc}{
    código = {1707008},
    nome = {Trabalho de Conclusão de Curso (TCC)},
    nome = {Trabalho de Conclusão de Curso},
    objetivo = {Conduzir o estudante à elaboração de um trabalho de pesquisa original, integrando conhecimentos teóricos, metodológicos e práticos do curso.},
    ementa = {Desenvolvimento do projeto de pesquisa. Orientação individual. Redação e formatação de TCC. Apresentação e defesa pública. Normas acadêmicas e avaliação científica.},
    pré-requisitos = {ai-proj2},
    horas teóricas = 180,
    horas práticas = 0,
    horas extensionistas = 0,
    horas estágio = 0,
    departamento = DF,
    bibliografia básica = {Rodrigues, T. \textit{Como Elaborar um TCC}. Lúmen, 2019.
    \newline Lima, P. \textit{Pesquisa e Escrita Científica}. Arcádia, 2021.
    \newline Melo, R. \textit{Orientação e Projeto}. Thalassa, 2020.
    \newline Castro, A. \textit{Do Projeto à Defesa}. Solaris, 2018.},
    bibliografia complementar = {Dias, M. \textit{Guia de Redação Científica}. Arqué, 2020.
    \newline Nogueira, L. \textit{TCC nas Humanidades}. Ânfora, 2021.
    \newline Brito, A. \textit{Pesquisa Autônoma}. Herma, 2021.},
    competências = {Elaborar pesquisa científica independente; redigir e apresentar TCC; aplicar princípios metodológicos e éticos em investigação acadêmica.},
}

\PPCDefinaAtividade{ai-estagio}{
    código = {1707009},
    nome = {Estágio Supervisionado ou Projeto de Extensão Aplicado},
    nome abreviado = {Estágio Superv. ou Proj. de Ext. Aplicado},
    objetivo = {Proporcionar experiência prática de aplicação dos saberes do curso em contextos profissionais, educativos e culturais.},
    ementa = {Vivência supervisionada em instituições culturais, educacionais ou de pesquisa. Desenvolvimento de projetos aplicados. Reflexão crítica sobre a prática profissional e o papel social das Humanidades.},
    pré-requisitos = {ai-proj2},
    horas teóricas = 0,
    horas práticas = 0,
    horas extensionistas = 0,
    horas estágio = 180,
    departamento = DH,
    bibliografia básica = {Melo, R. \textit{Extensão e Experiência}. Thalassa, 2020.
    \newline Rodrigues, T. \textit{Humanidades em Ação}. Lúmen, 2021.
    \newline Lima, P. \textit{Prática e Reflexão}. Arcádia, 2019.
    \newline Castro, A. \textit{Saberes Aplicados}. Solaris, 2018.},
    bibliografia complementar = {Dias, M. \textit{Estágio e Formação Humana}. Arqué, 2020.
    \newline Nogueira, L. \textit{Aprender Fazendo}. Ânfora, 2021.
    \newline Brito, A. \textit{Educação e Cultura em Prática}. Herma, 2020.},
    competências = {Aplicar conhecimentos teóricos em situações práticas; desenvolver atividades culturais e educativas; refletir sobre o papel social do saber humanístico.},
}
