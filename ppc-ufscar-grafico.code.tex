%! Author = Jander Moreira
%! Date = 11/10/2025
%! Package = ppc-ufscar

%%%%%%%%%%%%%%%%%%%%%%%%%%%%%%%%%%%%%%%%%%%%%%%%%%%%%%%%%%%
%% DESENHO GRÁFICO

\RequirePackage{etoolbox}
\RequirePackage{tikz}
\usetikzlibrary{calc, positioning, fit}

% \ppc@BaselineSkip é definido no ppc-ufscar.cls; se for usado apenas o
%   pacote (.sty), ele não existe
\ifdeflength{\ppc@BaselineSkip}{}{
    \newlength{\ppc@BaselineSkip}
    \setlength{\ppc@BaselineSkip}{\baselineskip}
}


% PPCGraficos: define as opções para os desenhos gráficos
% #1: lista de opções separadas por vírgulas (chave = valor)
\ExplSyntaxOn
\keys_define:nn { graficos } {
    largura~caixa .dim_set:N = \ppc@LarguraCaixaDisciplina,
    altura~caixa .dim_set:N = \ppc@AlturaCaixaDisciplina,
    altura~nome~área .dim_set:N = \ppc@AlturaNomeArea,
    separação .dim_set:N = \ppc@SeparacaoCaixasDisciplina,
    cor~fundo~obrigatória .code:n = {\colorlet{cor~caixa~obrigatória}{#1}},
    cor~texto~obrigatória .code:n = {\colorlet{cor~texto~obrigatória}{#1}},
    cor~horas~obrigatória .code:n = {\colorlet{cor~horas~obrigatória}{#1}},
    cor~gradiente~obrigatória.code:n = {\colorlet{cor~gradiente~obrigatória}{#1}},
    cor~fundo~optativa .code:n = {\colorlet{cor~caixa~optativa}{#1}},
    cor~texto~optativa .code:n = {\colorlet{cor~texto~optativa}{#1}},
    cor~horas~optativa .code:n = {\colorlet{cor~horas~optativa}{#1}},
    cor~gradiente~optativa.code:n = {\colorlet{cor~gradiente~optativa}{#1}},
    cor~fundo~área .code:n = {\colorlet{cor~fundo~área}{#1}},
    cor~texto~período .code:n = {\colorlet{cor~texto~período}{#1}},
    cor~fundo~período .code:n = {\colorlet{cor~fundo~período}{#1}},
    fonte~texto .tl_set:N = \ppc@FonteTexto,
    fonte~horas .tl_set:N = \ppc@FonteHoras,
    fonte~período .tl_set:N = \ppc@FontePeriodo,
    mostra~total~disciplina .bool_set:N = \ppc@MostraTotal,
    mostra~nome~área .bool_set:N = \ppc@MostraNomeArea,
    mostra~coluna~períodos .bool_set:N = \ppc@MostraColunaPeriodos,
    visão~do~aluno .bool_set:N = \ppc@VisaoDoAluno,
}
\NewDocumentCommand{\PPCGraficos}{ >{ \TrimSpaces } m }{
    \keys_set:nn { graficos } { #1 }
}
\ExplSyntaxOff


% Configuração padrão para os gráficos
\PPCGraficos{
    largura caixa = 4cm,
    altura caixa = 2.5cm,
    altura nome área = 1.5cm,
    separação = 0.2cm,
    cor fundo obrigatória = blue!40,
    cor texto obrigatória = black,
    cor horas obrigatória = black,
    cor gradiente obrigatória = blue!10,
    cor fundo optativa = teal!40,
    cor texto optativa = black,
    cor horas optativa = black,
    cor gradiente optativa = teal!10,
    cor fundo área = black!10,
    cor texto período = blue!20!black,
    cor fundo período = cyan!80!black!40,
    fonte texto = \sffamily\bfseries\small,
    fonte horas = \sffamily\bfseries\small,
    fonte período = \sffamily\bfseries\LARGE,
    mostra total disciplina,
    mostra nome área,
    mostra coluna períodos,
    visão do aluno = false,
}

\tikzset{
    nó limpo/.style = {
        draw = none,
        inner sep = 0pt,
        outer sep = 0pt,
        anchor = center,
    },
    todos nós/.style = {
        execute at begin node=\setlength{\baselineskip}{\ppc@BaselineSkip},
        inner sep = 0.15cm,
        outer sep = 0pt,
        anchor = center,
    },
    caixa disciplina/.style = {
        todos nós,
        inner sep = 0pt,
        fill = cor caixa,
        text width = \ppc@LarguraCaixaDisciplina,
        text height = \ppc@AlturaCaixaDisciplina,
        anchor = center,
    },
    caixa período/.style = {
        nó limpo,
        todos nós,
        align = center,
        inner sep = 0pt,
        fill = cor fundo período,
        text width = 0.5\ppc@LarguraCaixaDisciplina,
        text height = \ppc@AlturaCaixaDisciplina,
        anchor = center,
    },
    nome disciplina/.style = {
        todos nós,
        anchor = center,
        align = center,
        text width = \dimexpr 0.95\ppc@LarguraCaixaDisciplina - 0.3cm, % -2 * innersep
        font = \ppc@FonteTexto,
        text = cor texto,
    },
    texto período/.style = {
        font = \ppc@FontePeriodo,
        text = cor texto período,
    },
    horas/.style = {
        todos nós,
        anchor = south west,
        align = center,
        % rounded corners = 0.25cm,
        % draw = cor caixa,
        text width = \dimexpr 0.25\ppc@LarguraCaixaDisciplina - 0.3cm,
        font = \ppc@FonteHoras,
        text = cor horas,
    },
    prefixo/.style = {
        todos nós,
        inner sep = 0.1pt,
        anchor = north west,
        text = cor horas,
        font = \ppc@FonteTexto\tiny,
    },
    teóricas/.style = {fill = cor caixa!80!cor gradiente},
    práticas/.style = {fill = cor caixa!60!cor gradiente},
    extensionistas/.style = {fill = cor caixa!40!cor gradiente},
    estágio/.style = {fill = cor caixa!20!cor gradiente},
    nome área/.style = {
        font = \ppc@FonteTexto\small,
        fill = cor fundo área,
        anchor = center,
        align = center,
    },
    caixa de área/.style = {
        todos nós,
        anchor = north west,
        inner sep = 0pt,
        % inner sep = \ppc@SeparacaoCaixasDisciplinas,
    },
}

\PPCDefinaDisciplina{ppc@optativa-fantasma}{
    código = {},
    nome = {Optativa},
    objetivo = {},
    ementa = {},
    caráter = Optativo,
    pré-requisitos = {},
    horas teóricas = 60,
    horas práticas = 0,
    horas extensionistas = 0,
    horas estágio = 0,
    departamento = {},
    bibliografia básica = {},
    bibliografia complementar = {},
    competências = {},
}


% ppc@CaixaDisciplina: desenha a caixa da disciplina
% #1: id da disciplina
\NewDocumentCommand{\ppc@CaixaDisciplina}{ m }{%
    \ATAttributeIfEqual[#1]{caráter}{Obrigatório}{%
        \colorlet{cor caixa}{cor caixa obrigatória}%
        \colorlet{cor texto}{cor texto obrigatória}%
        \colorlet{cor horas}{cor horas obrigatória}%
        \colorlet{cor gradiente}{cor gradiente obrigatória}%
    }{%
        \ATAttributeIfEqual[#1]{caráter}{Optativo}{%
            \colorlet{cor caixa}{cor caixa optativa}%
            \colorlet{cor texto}{cor texto optativa}%
            \colorlet{cor horas}{cor horas optativa}%
            \colorlet{cor gradiente}{cor gradiente optativa}%
        }{%
            \ATAttributeGetTo[#1]{\ppg@local@Carater}{caráter}
            \PackageError{ppc-ufscar}{Caráter inválido para #1 (\ppg@local@Carater); válidos: obrigatório ou optativo.}
        }%
    }%
    \begin{tikzpicture}
        \node[caixa disciplina] (caixa disciplina) {};
        \node[nome disciplina] at ($(caixa disciplina.south)!0.65!(caixa disciplina.north)$) (nome disciplina) { \ATAttributeIfExist[#1]{nome abreviado}{\PPCDisciplina{#1}{nome abreviado}}{\PPCDisciplina{#1}{nome}}};
        \coordinate (horas) at ($(caixa disciplina.south west) + (0, 0.05cm)$);
        \foreach \prefixo/\horas in {
            T:/teóricas, P:/práticas, Ext:/extensionistas, Est:/estágio%
        }{
            \node[horas, \horas] at (horas) (ultima hora) {\PPCDisciplina{#1}{horas \horas}};
            \node[prefixo] at (ultima hora.north west) {\prefixo};
            \coordinate (horas) at (ultima hora.south east);
        }
        \ppc@ObtenhaCargaDisciplina{\ppc@local@Carga}{#1}
        \PPCIf{ppc@MostraTotal}{
            \node[horas, anchor = south west, text = cor texto, scale = 0.95, align = right, inner xsep = 0.1cm] at (ultima hora.north west) (ultima hora) {\ppc@local@Carga h};
        }{}
        % \node[prefixo] at (ultima hora.north west) {Tot:};
    \end{tikzpicture}%
}

% ppc@CaixaPeriodo: desenha a caixa resumo do período
% #1: (opcional) nome da matriz
% #2: número do período
\NewDocumentCommand{\ppc@CaixaPeriodo}{ O {matriz-principal} m }{%
    \begin{tikzpicture}
        \node[caixa período] (caixa período) {};
        \node[nó limpo, texto período] at ($(caixa período)!0.5!(caixa período.north)$) {#2};
        \node[nó limpo, texto período] at ($(caixa período)!0.5!(caixa período.south)$) {\small\PPCCargaHoraria*[#1]{total #2}[\,h]};
    \end{tikzpicture}%
}


% ppc@CaixaArea: desenha a caixa para uma área específica, incluindo todos os períodos
% #1: (opcional) nome da matriz
% #2: id da área
\newcounter{ppc@local@Linha}
\newcounter{ppc@local@Coluna}
\newcounter{ppc@local@Maximo}
\NewDocumentCommand{\ppc@CaixaArea}{ O{matriz-principal} m }{%
    \begin{tikzpicture}
        \pgfdeclarelayer{disciplinas}
        \pgfdeclarelayer{áreas}
        \pgfsetlayers{áreas, disciplinas, main}

        % Disciplinas
        \begin{pgfonlayer}{disciplinas}
            \defcounter{ppc@local@Maximo}{0}
            \defcounter{ppc@local@Linha}{0}
            \providebool{ppc@ApresenteOptativa}
            \PPCParaCadaPeriodo[#1]{\periodo}{
                \defcounter{ppc@local@Coluna}{0}
                \ATListForEach{#1-#2-\periodo}{\disciplina}{
                    \ifstrequal{#2}{periodo}{
                        \ATAttributeIfEqual[\disciplina]{caráter}{Obrigatório}{
                            \booltrue{ppc@ApresenteOptativa}
                        }{
                            \boolfalse{ppc@ApresenteOptativa}
                        }
                    }{
                        \booltrue{ppc@ApresenteOptativa}
                    }
                    \ifbool{ppc@ApresenteOptativa}{
                        \node[nó limpo] at (\theppc@local@Coluna * \ppc@LarguraCaixaDisciplina + \theppc@local@Coluna * \ppc@SeparacaoCaixasDisciplina, \theppc@local@Linha * -\ppc@AlturaCaixaDisciplina - \theppc@local@Linha * \ppc@SeparacaoCaixasDisciplina
                        ) {\ppc@CaixaDisciplina{\disciplina}};
                        \global\stepcounter{ppc@local@Coluna}
                    }{}
                }
                \ifstrequal{#2}{periodo}{
                    \ATListForEach{#1-carga-optativas-\periodo}{\cargahoraria}{
                        \PPCDefinaDisciplina{ppc@optativa-fantasma}{
                            horas teóricas = \cargahoraria,
                        }
                        \node[nó limpo] at (\theppc@local@Coluna * \ppc@LarguraCaixaDisciplina + \theppc@local@Coluna * \ppc@SeparacaoCaixasDisciplina, \theppc@local@Linha * -\ppc@AlturaCaixaDisciplina - \theppc@local@Linha * \ppc@SeparacaoCaixasDisciplina
                        ) {\ppc@CaixaDisciplina{ppc@optativa-fantasma}};
                        \global\stepcounter{ppc@local@Coluna}
                    }
                }{}
                \ifnumcomp{\theppc@local@Coluna}{>}{\theppc@local@Maximo}{
                    \global\defcounter{ppc@local@Maximo}{\theppc@local@Coluna}
                }{}
                \global\stepcounter{ppc@local@Linha}
            }
            \global\addtocounter{ppc@local@Linha}{-1} % volta 1 na contagem
        \end{pgfonlayer}

        \begin{pgfonlayer}{áreas}
            % caixa com as disciplinas
            \coordinate (caixa area sup esq) at (-\ppc@LarguraCaixaDisciplina/2 - \ppc@SeparacaoCaixasDisciplina, \ppc@AlturaCaixaDisciplina/2 + \ppc@SeparacaoCaixasDisciplina);
            \coordinate (caixa area inf dir) at (\theppc@local@Maximo * \ppc@LarguraCaixaDisciplina - \ppc@LarguraCaixaDisciplina/2 + \theppc@local@Maximo * \ppc@SeparacaoCaixasDisciplina, -\theppc@local@Linha * \ppc@AlturaCaixaDisciplina - \theppc@local@Linha * \ppc@SeparacaoCaixasDisciplina - \ppc@AlturaCaixaDisciplina/2 - \ppc@SeparacaoCaixasDisciplina);
            \node[nó limpo, nome área, fit = (caixa area sup esq)(caixa area inf dir)] (caixa area) {};

            % caixa com o nome da área
            \PPCIf{ppc@MostraNomeArea}{
                \coordinate (caixa area inf esq) at (-\ppc@LarguraCaixaDisciplina/2 - \ppc@SeparacaoCaixasDisciplina, \ppc@AlturaCaixaDisciplina/2 + 2\ppc@SeparacaoCaixasDisciplina);
                \coordinate (caixa area sup dir) at (\theppc@local@Maximo * \ppc@LarguraCaixaDisciplina - \ppc@LarguraCaixaDisciplina/2 + \theppc@local@Maximo * \ppc@SeparacaoCaixasDisciplina, \ppc@AlturaNomeArea + \ppc@AlturaCaixaDisciplina/2 + 2\ppc@SeparacaoCaixasDisciplina);
                \node[nó limpo, nome área, fit=(caixa area inf esq)(caixa area sup dir)] (título) {};
                \node[nome área, inner sep = 0.25cm, text width = \theppc@local@Maximo * \ppc@LarguraCaixaDisciplina + \theppc@local@Maximo * \ppc@SeparacaoCaixasDisciplina - 3\ppc@SeparacaoCaixasDisciplina] at (título) {\PPCArea[#1]{#2}{nome}};
            }{}
        \end{pgfonlayer}
    \end{tikzpicture}%
}


% ppc@CaixaPeriodos: desenha a caixa a coluna de períodos
% #1: (opcional) nome da matriz
\NewDocumentCommand{\ppc@CaixaPeriodos}{ O{matriz-principal} }{%
    \begin{tikzpicture}
        \pgfdeclarelayer{períodos}
        \pgfdeclarelayer{coluna}
        \pgfsetlayers{coluna, períodos, main}

        % Períodos
        \begin{pgfonlayer}{períodos}
            \defcounter{ppc@local@Linha}{0}
            \PPCParaCadaPeriodo[#1]{\periodo}{
                \node[nó limpo] at (0, \theppc@local@Linha * -\ppc@AlturaCaixaDisciplina - \theppc@local@Linha * \ppc@SeparacaoCaixasDisciplina) {
                    \ppc@CaixaPeriodo[#1]{\periodo}
                };
                \global\stepcounter{ppc@local@Linha}
            }
            \global\addtocounter{ppc@local@Linha}{-1} % volta 1 na contagem
        \end{pgfonlayer}

        \begin{pgfonlayer}{coluna}
            % caixa com os períodos
            \coordinate (caixa area sup esq) at (-\ppc@LarguraCaixaDisciplina/4 - \ppc@SeparacaoCaixasDisciplina, \ppc@AlturaCaixaDisciplina/2 + \ppc@SeparacaoCaixasDisciplina);
            \coordinate (caixa area inf dir) at (\ppc@LarguraCaixaDisciplina/4 + \ppc@SeparacaoCaixasDisciplina, -\theppc@local@Linha * \ppc@AlturaCaixaDisciplina - \theppc@local@Linha * \ppc@SeparacaoCaixasDisciplina - \ppc@AlturaCaixaDisciplina/2 - \ppc@SeparacaoCaixasDisciplina);
            \node[nó limpo, nome área, fit = (caixa area sup esq)(caixa area inf dir)] (caixa area) {};

            \PPCIf{ppc@MostraNomeArea}{
                \coordinate (caixa area inf esq) at (-\ppc@LarguraCaixaDisciplina/4 - \ppc@SeparacaoCaixasDisciplina, \ppc@AlturaCaixaDisciplina/2 + 2\ppc@SeparacaoCaixasDisciplina);
                \coordinate (caixa area sup dir) at (\ppc@LarguraCaixaDisciplina/4 + \ppc@SeparacaoCaixasDisciplina, \ppc@AlturaNomeArea + \ppc@AlturaCaixaDisciplina/2 + 2\ppc@SeparacaoCaixasDisciplina);
                \node[nó limpo, nome área, fit=(caixa area inf esq)(caixa area sup dir)] (título) {};
                \node[nome área, inner sep = 0.25cm, text width = \ppc@LarguraCaixaDisciplina/2 - 3\ppc@SeparacaoCaixasDisciplina] at (título) {Total: \PPCCargaHoraria[#1]{total}[\,h]};
            }{}
        \end{pgfonlayer}
    \end{tikzpicture}%
}


% ppc@CaixaMatriz: desenha a caixa da matriz curricular completa
% #1: (opcional) nome da matriz
% #2: id da área
\NewDocumentCommand{\ppc@CaixaMatriz}{ O{matriz-principal} }{%
    \begin{tikzpicture}[transform shape, scale = 0.5]
        \PPCIf{ppc@MostraColunaPeriodos}{
            \node[caixa de área] (períodos) {\ppc@CaixaPeriodos[#1]};
            \coordinate (referência) at ($(períodos.north east) + (\ppc@SeparacaoCaixasDisciplina, 0)$);
        }{
            \coordinate (referência) at (0, 0);
        }
        \PPCIf{ppc@VisaoDoAluno}{
            \node[caixa de área] at (referência) (área) {\ppc@CaixaArea[#1]{periodo}};
        }{
            \ATListForEach{#1-áreas}{\area}{
                \node[caixa de área] at (referência) (área) {\ppc@CaixaArea[#1]{\area}};
                \coordinate (referência) at ($(área.north east) + (\ppc@SeparacaoCaixasDisciplina, 0)$);
            }
        }
    \end{tikzpicture}
}

\NewDocumentCommand{\PPCMatrizCurricular}{ O{matriz-principal} D<>{} }{%
    \begingroup
    \PPCGraficos{#2}%
    \ppc@CaixaMatriz[#1]%
    \endgroup
}