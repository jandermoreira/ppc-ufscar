%! Author = Jander Moreira (moreira.jander@gmail.com)
%! Date = 25/04/2025

\documentclass[a4paper, 11pt]{article}
\usepackage[T1]{fontenc}
\usepackage[utf8]{inputenc}
\usepackage[brazilian]{babel}

\usepackage[presets]{packdoc}
\usepackage{ppc-ufscar}
\usepackage{hvlogos}

\title{
    A classe \PackageName{ppc-ufscar}\\
    \small V.\PPCVersion\\
    \url{https://github.com/jandermoreira/ppc-ufscar}
}
\author{Jander Moreira\\\footnotesize\texttt{moreira.jander@ufscar.br}}
\date{}

\begin{document}
\maketitle

\PDNewVersion{0.1}{2025-04-25}
\PDAddChange{0.1}{
    description = Versão inicial.,
    no page,
    no box,
}
\PDPrintChanges


\section{Introdução}
Esta classe é o resultado do trabalho realizado nas reformulações dos cursos de Bacharelado em Engenharia da Computação e Bacharelado da Ciência da Computação em 2024. Durante a elaboração do documento, um mecanismo mais robusto para ``automatizar'' certas partes da escrita foram necessários.


\section{Como usar}
Para usar a classe, o documento deve especificar \PackageName{ppc-ufscar} em \Macro{documentclass}.

\begin{PDListing}
    \documentclass{ppc-ufscar}
\end{PDListing}

A classe é desenvolvida sobre a classe padrão \PackageName{report}. Alternativamente, parte dos recursos podem ser usados em outras classes e devem ser carragados como um pacote.

\begin{PDListing}
    \documentclass[a4paper, 11pt]{article}  % por exemplo
    \usepackage{ppc-ufscar}
\end{PDListing}

Quando carregado como pacote, os ajustes relativos ao leiaute não são usados, como margens e espaçamento de linhas, por exemplo.

A classe será disponibilizada no CTAN\footnote{\url{https://ctan.org}.} quando atingir uma maturidade mínima.


\section{Recursos para o PPC}

\subsection{Capa: nome, autoria e data}

\subsection{Especificação da ficha de informações das disciplinas}

Como apresentar uma disciplina.

\subsection{Especificação da grade curricular}


\section{Recursos de compilação condicional}
\index{Overleaf}%
\index{timeout@\textit{timeout}}%
Os projetos que usam a classe \PackageName{ppc-ufscar} tendem usar uma quantidade relativamente alta de recursos do \LaTeX\ e também aumentam o tempo de compilação. Em particular, quando usado no Overleaf\footnote{\url{https://www.overleaf.com}.}, é comum que a compilação falhe por \textit{timeout} (reduzido para 20~segundos para as contas grátis em 2024).


\section{Descrição de recursos de uso geral}

\begin{Macrodef}{NovoAtributo}{\MArg{classe}\MArg{nome}\MArg{valor}}{}

\end{Macrodef}

\begin{Macrodef}{Atributo}{\MArg{classe}\MArg{nome}}{}

\end{Macrodef}


\section{Ainda por fazer}
Esta é uma \textit{lista de desejos} que podem ou não ser concretizados no furturo:

\begin{itemize}
    \item Substituição do texto usado nas bibliografias das disciplinas por uma lista de rótulos \BibTeX\ para formatação automática segundo a ABNT.
    \item Geração de grade de disciplinas com conexões representando pré-requisitos.
\end{itemize}

\printindex

\end{document}
