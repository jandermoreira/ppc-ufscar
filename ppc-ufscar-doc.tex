%! Author = Jander Moreira (moreira.jander@gmail.com)
%! Date = 25/04/2025

\documentclass[a4paper, 11pt]{article}
\usepackage[T1]{fontenc}
\usepackage[utf8]{inputenc}
\usepackage[brazilian]{babel}

\usepackage[presets]{packdoc}
\PDSetElement{Macro}{
    index heading = Todos os comandos,
    index remark = {~(comando)},
% no single index
}
\PDSetElement{Option}{
    index heading = Todas as opções,
    index remark = {~(opção)},
% no single index
}
\PDSetElement{Environment}{
    index heading = Todos os ambientes,
    index remark = {~(ambiente)},
% no single index
}
\usepackage{ppc-ufscar}
\usepackage{hvlogos}
\usepackage{todonotes}

\title{
    A classe \PackageName{ppc-ufscar}\\
    \small V.\PPCVersion\\
    \url{https://github.com/jandermoreira/ppc-ufscar}
}
\author{Jander Moreira\\\footnotesize\texttt{moreira.jander@ufscar.br}}
\date{}

\begin{document}
\maketitle

\PDNewVersion{0.1}{2025-04-25}
\PDAddChange{0.1}{
    description = Versão inicial.,
    no page,
    no box,
}
\PDPrintChanges


\section{Introdução}
Esta classe é o resultado do trabalho realizado nas reformulações dos cursos de Bacharelado em Engenharia da Computação e Bacharelado da Ciência da Computação em 2024. Durante a elaboração do documento, um mecanismo mais robusto para ``automatizar'' certas partes da escrita foram necessários.

Ao se escrever um PPC, muitas informações são repetidas à exaustão: o nome de uma disciplina aparece na grade curricular, na listagem de disciplinas por período e na lista de ementas, entre outros locais. Além disso, ``pequenas'' alterações, como mudar uma disciplina de período gera implicações em diversas outras partes do documento, como no quadro de disciplinas por período, total de horas do período etc.

As ferramentas disponibilizadas aqui têm, portanto, o objetivo de facilitar essa organização, de forma que as alterações sejam feitas em um só local e seus efeitos sejam propagados para onde forem necessários.


\section{Como usar a classe ou o pacote}
Para usar a classe, o documento deve especificar \PackageName{ppc-ufscar} em \Macro{documentclass}.

\begin{PDListing}
    \documentclass{ppc-ufscar}
\end{PDListing}

A classe é desenvolvida sobre a classe padrão \PackageName{report}. Alternativamente, parte dos recursos podem ser usados em outras classes e devem ser carragados como um pacote.

\begin{PDListing}
    \documentclass[a4paper, 11pt]{article}  % por exemplo
    \usepackage{ppc-ufscar}
\end{PDListing}

Quando carregado como pacote, os ajustes relativos ao leiaute não são usados, como margens e espaçamento de linhas, por exemplo.

A classe será disponibilizada no CTAN\footnote{\url{https://ctan.org}.} quando atingir uma maturidade mínima.


\section{A filosofia no uso das ferramentas}
Todo PPC tem vários elementos em comum, como capa, descrição de dados institucionais, ficha descritiva do curso, matriz curricular (grade de disciplinas), informações sobre a disciplinas (código, nome, ementa etc.) e muitos outros.

Os elementos de suporte à organização são apresentados na \Cref{tab:elementos}.

\begin{table}[H]
    \caption{Elementos de organização do PPC.}
    \label{tab:elementos}
    \centering
    \begin{tabular}{>{\raggedright\arraybackslash}p{0.25\linewidth}>{\RaggedRight\arraybackslash}p{0.65\linewidth}}
        \hfill\textbf{Elemento}\hfill\null & \hfill\textbf{Descrição}\hfill\null \\
        \cmidrule(lr){1-1} \cmidrule(lr){2-2} \\
        Ficha descritiva do curso &
        Na ficha descritiva são especificados os dados do curso, como nome da instituição, campus, centro, modalidade, número de vagas e carga horária, entre outros. O comando \MacroRef{PPCDefinaFichaDescritiva} é usado para esta especificação. Essas informações são usadas para gerar a capa e a ficha descritiva do curso. (\Cref{sec:ficha-descritiva-do-curso})
        \\
        Dados institucionais &
        Neste contexto, os dados institucionais se referem às informações das pessoas envolvidas: nome do reitor, do pró-reitor de graduação, diretor de centro, coordenador de curso e membros do NDE, por exemplo. Para isso é usado o comando \Macro{PPCDefinaDadosInstitucionais} e pode ser gerada a página com essa listagem. (\Cref{sec:apresentacao-institucional})
        \\
        Descrição das disciplinas &
        Cada disciplina deve ter sua descrição completa, o que é feito com o comando \Macro{PPCDefinaDisciplina}. Por padrão, cada disciplina é definida em um arquivo separado (facilita a organização, a edição e controle de alterações), mas nada impede que as definições sejam feitas conforme as necessidades de quem escreve o documento. Se postas em arquivos, elas podem ser importadas para o documento principal com \Macro{PPCImporteDisciplinas}. Com essas informações as fichas de cada disciplina podem ser geradas. Essa definição tem que, necessariamente, ser feita antes da definição da matriz curricular. (\Cref{sec:definicao-de-disciplina})
        \\
        Matriz curricular &
        A definição da matriz corresponde à organização das disciplinas no curso. Permite especificar as disciplinas por área, além de indicar a carga de optativas e listar possíveis eletivas. Usando \Macro{PPCDefinaMatrizCurricular}, são listadas as disciplinas período por período e indicando quais são obrigatórias e quais são optativas. Essa definição permite saber o número de períodos, quais são as disciplinas do curso e seu caráter, quais são as áreas, determinar a carga horária total e por período. A partir dela é gerada o elemento gráfico grade curricular e as listagens de disciplinas por período. \\
    \end{tabular}
\end{table}


\section{Descrição das ferramentas}
Considerando-se a parte prática do uso do pacote, nesta seção é descrito como usar os recursos essenciais para a criação do PPC.

\subsection{Ficha descritiva do curso}\label{sec:ficha-descritiva-do-curso}
Os dados do curso são definidos pelo comando \Macro{PPCDefinaFichaDescritiva}. Esses dados são usado para gerar tanto a capa do documento quanto a ficha do curso.

\begin{Macrodef}{PPCDefinaFichaDescritiva}{\MArg{lista-de-atributos}}{}
    Este comando é usado para especificar os diversos atributos do curso. Há uma lista de atributos obrigatórios; outros tantos podem ser opcionalmente incluídos.

    Cada atributo é sempre indicado por um par \MArg{título}\MArg{valor}. Os títulos são usados na ficha descritiva.
\end{Macrodef}


Em particular, os atributos \PDInline{ano} e \PDInline{local} são usados apenas na capa e são omitidos na ficha descritiva do curso.

Manualmente, a apresentação da ficha pode ser feita com comando \MacroRef{PPCFichaDescritiva}. Cada atributos pode ser usado com \MacroRef{PPCInfo}: o valor é recuperado usando \Macro{PPCInfo}\PArg{atributo} e o título usando-se \Macro{PPCInfo}\PArg{texto atributo}.

Ficha de caracterização: atributos já com texto descritivo: \ATForEach{\Atributo}{
    universidade, curso, campus, centro, ano, modalidade, vagas,
    turno, carga horária, regime, duração, duração mínima, duração máxima,
    ato legal, última reformulação, local
}[{ e }{, }{ e }{.}]{\mbox{\texttt{\Atributo}}}

Segue um exemplo.

\begin{PDExample}
    \PPCDefinaFichaDescritiva{
        instituição = {Instituição de ensino}{Universidade Federal de São Carlos},
        campus = {Campus}{São Carlos},
        centro = {Centro}{Centro de Ciências Aleatórias},
        curso = {Nome do curso}{Bacharelado em Alguma Coisa},
        sigla = {Sigla do curso}{AC},
        modalidade = {Modalidade}{Presencial/Híbrido},
        vagas = {Quantidade de vagas}{60 vagas anuais},
        turno = {Turno}{Integral},
        carga horária = {Carga horária total}{3240 horas},
        regime = {Regime}{Semestral},
        duração = {Tempo de duração}{8 semestres},
        duração mínima = {Tempo de duração mínima}{6 semestres},
        duração máxima = {Tempo de duração máxima}{14 semestres},
        ato legal = {Ato legal}{Parecer 1522/79 de 11 de novembro de 1979},
        última reformulação = {Ano da última reformulação}{2018},
        ano = {Ano}{2025},
        local = {Local}{São Carlos -- SP},
    }

    \textbf{\PPCInfo{curso}}:\par
    \PPCInfo{texto carga horária} de \PPCInfo{carga horária}.
\end{PDExample}

A criação de atributos personalizados é livre, sendo que eles serão incluídos na ficha descritiva.

Alguns dos atributos são usados para gerar a capa e a ficha descritiva do curso automaticamente. Esses campos, nesta situação, ficam ``obrigatórios''.

\subsection{Apresentação institucional}\label{sec:apresentacao-institucional}
A dados institucionais (ou ``segunda página'') é descrita com \MacroRef{PPCDefinaDadosInstitucionais}.

\begin{Macrodef}{PPCDefinaDadosInstitucionais}{}{}
    Define o conteúdo da dados institucionais. A geração automática da página com a descrição segue a ordem em que são especificadas cada \PDInline{unidade}, \PDInline{cargo} e \PDInline{pessoa}. Adicionalmente, uma mudança de página pode ser adicionada manualmente usando-se \PDInline{nova página}.
\end{Macrodef}

\begin{PDListing}
    \PPCDefinaDadosInstitucionais{
        unidade = {\PPCInfo{instituição}},
        cargo = {Reitora},
        pessoa = {Profa. Dra. Zargulinda Mevânia},
        cargo = {Vice-Reitor},
        pessoa = {Prof. Dr. Quiríntio Marcus Bolvéria},
        cargo = {Pró-Reitor de Graduação},
        pessoa = {Prof. Dr. Tarsílio Vondrake},
        unidade = {\PPCInfo{centro}},
        cargo = {Diretor},
        pessoa = {Prof. Dr. Elvírio Crantobel},
        cargo = {Vice-Diretor},
        pessoa = {Prof. Dr. Mandrílio Estévano},
        unidade = {Coordenação de Curso de \PPCInfo{curso}},
        cargo = {Coordenadora},
        pessoa = {Profa. Dra. Ninfária Clotilde},
        cargo = {Vice-Coordenador},
        pessoa = {Prof. Dr. Valtrônio Mirgélio},
        cargo = {Secretária},
        pessoa = {Sra. Dulmira Xantofreda},
        nova página,
        unidade = {Núcleo Docente Estruturante},
        cargo = {Presidente},
        pessoa = {Prof. Dr. Grivaldo Ortênsio},
        cargo = {Membros},
        pessoa = {Prof. Dr. Jandríaco Fluvério},
        pessoa = {Profa. Dra. Minésia Cravélia},
        pessoa = {Prof. Dr. Odirlânio Tebaldo},
        pessoa = {Profa. Dra. Flandina Quelúria},
        unidade = {Outros colaboradores},
        pessoa = {Sra. Vólmira Trevíssima},
    }
\end{PDListing}

A apresentação da ficha pode ser feita manualmente com \MacroRef{PPCDadosInstitucionais}.

\subsection{Definição de disciplinas}\label{sec:definicao-de-disciplina}

Cada disciplina deve ser especificada usando-se \Macro{PPCDefinaDisciplina}. O primeiro parâmetro é o identificador da disciplina, usualmente um apelido comum, como \PDInline{tcc} para \textit{Trabalho de Conclusão de Curso}, por exemplo.

\begin{Macrodef}{PPCDefinaDisciplina}{\MArg{id-disciplina}\MArg{lista-atributos}}{}
    Define uma disciplina com identificador \Argument{id-disciplina} e usando \Argument{lista-atributos} para determinar os valores de cada atributo.

\end{Macrodef}

\begin{PDExample}
    \PPCDefinaDisciplina{cálculo1}{
        nome = {Cálculo Diferencial e Integral 1},
        pré-requisitos = {algelin},
        horas teóricas = 60,
        horas práticas = 30,
        horas extensionistas = 0,
        horas estágio = 0,
    }
    A disciplina \PPCDisciplina{cálculo1}{nome} possui carga horária total de \PPCDisciplina{cálculo1}{horas total}\,h.
\end{PDExample}

Alguns atributos merecem destaque, sendo apresentados na \Cref{tab:atributos-disciplinas}.

\begin{table}
    \caption{Atributos de disciplinas que merecem destaque.}
    \label{tab:atributos-disciplinas}
    \centering
    \begin{tabular}{>{\raggedright\arraybackslash}p{0.35\linewidth}>{\RaggedRight\arraybackslash}p{0.55\linewidth}}
        \hfill\textbf{Atributos}\hfill\null & \hfill\textbf{Considerações}\hfill\null \\
        \cmidrule(lr){1-1} \cmidrule(lr){2-2}
        \PDInline{caráter} &
        Este atributo pode ter os valores \PDInline{Obrigatório}, \PDInline{Optativo} ou \PDInline{Eletivo}, sendo automaticamente atribuído ao se criar a matriz curricular com \Macro{PPCDefinaMatrizCurricular}, conforme detalhado na \Cref{sec:especificacao-da-matriz-curricular}.
        \\
        \PDInline{pré-requisitos} &
        A lista de pré-requisitos de uma disciplina pode ser vazia ou uma lista separada por vírgulas com um ou mais \Argument{id-disciplina}. Ao serem apresentados os quadros com as listas de disciplina por período (\MacroRef{PPCQuadroPeriodo}), essa lista é expandida para os nomes de cada disciplina.
        \\
        \PDInline{nome abreviado} &
        Esse atributo é opcional e pode ser omitido. Quando existir, ele é usado no nome da disciplina na apresentação gráfica da matriz curricular (\MacroRef{PPCMatrizCurricular}). Assim, nomes mais longos de disciplinas podem ter uma versão mais curta para que caibam nos espaços reservados. Siglas significativas também podem ser usadas, como TCC.
        \\
        \PDInline{horas teóricas}, \PDInline{horas práticas}, \PDInline{horas extensionistas} e \PDInline{horas estágio} &
        Esses atributos são obrigatórios e, a partir deles, o atributo \PDInline{horas total} é automaticamente calculado e incluído na descrição da disciplina. Terminada a definição de uma disciplina, \MacroRef{PPCDisciplina}\MArg{id}\PArg{horas total} pode ser usado.
        \\
    \end{tabular}
\end{table}

\subsection{Especificação da matriz curricular}\label{sec:especificacao-da-matriz-curricular}
A matriz curricular, feita com \MacroRef{PPCDefinaMatrizCurricular}, é a forma de organizar o conjunto de disciplinas.

Segue um exemplo de matriz, o qual será abordado por partes.

\begin{PDListing}
    \PPCDefinaMatrizCurricular{
        nome período = {trimestre},
        número períodos = 8,
        área = {geo}{
            nome = {Ciências da Terra},
            1 = {geo-fund, metod-cient},
            2 = {mineral-bas, calc-elem},
            3 = {tecto-elem, fis-elem},
            4 = {petro-ign, estat-geo},
            5 = {geo-fis-apl},
            6 = {geo-morfo, *ped-bas},
            7 = {*explo-min},
        },
        área = {piro}{
            nome = {Ciências do Fogo},
            1 = {termo-din-elem},
            2 = {combust-fund, vulcan-int},
            3 = {geo-term},
            4 = {plas-fis-int, *piro-meta},
            6 = {*energ-renov, *energ-avanc},
        },
        área = {hidro}{
            nome = {Ciências da Água},
            2 = {hidro-ciclo, limno-bas},
            3 = {geo-quim-aq},
            5 = {sedimen-log, *crio-fund, *hidro-gest},
            7 = {ocean-fis},
        },
        área = {atmo}{
            nome = {Ciências do Ar},
            1 = {meteo-int},
            4 = {climato-glob, dinam-atmos, *sens-remoto},
            6 = {*micro-fis-nuv, *model-amb},
            8 = {*polu-ar, *geo-etica},
        },
        área = {estagio-tcc}{
            nome = {Estágio ou TCC},
            8 = {estagio, tcc},
        },
        carga optativas = {
            4 = {60},
            6 = {60, 60, 60},
            8 = {60, 60},
        },
        eletivas = {geo-hist, astro-elem},
        carga a compensar = {
            8 = 360,
        }
    }
\end{PDListing}

\subsubsection*{Nome e número de períodos}
O curso é organizado por períodos letivos. No exemplo é definido um curso de oito trimestres.

\begin{PDListing}
    nome período = {trimestre},
    número períodos = 8,
\end{PDListing}

O texto ``trimestre'' é usado na apresentação das tabelas de disciplinas por trimestre apenas.

\subsubsection*{Áreas}
O curso é organizado em quatro grandes áreas: Terra, Fogo, Água e Ar.

Para a área da água, seu identificador é \PDInline{hidro} e seu \PDInline{nome} é ``Ciências da Água''. Essa área não possui disciplinas nos trimestres 1, 4, 6 ou 8, mas tem duas disciplinas no segundo trimestre, uma no terceiro, três no quinto e uma no sétimo.

\begin{PDListing}
    área = {hidro}{
        nome = {Ciências da Água},
        2 = {hidro-ciclo, limno-bas},
        3 = {geo-quim-aq},
        5 = {sedimen-log, *crio-fund, *hidro-gest},
        7 = {ocean-fis},
    },
\end{PDListing}

\index{optativas}
\index{disciplinas!optativas}
\index{obrigatórias}
\index{disciplinas!obrigatórias}
Em particular, as disciplinas \PDInline{crio-fund} e \PDInline{hidro-geo} são marcadas como optativas, o que é indicado pelo asterisco que precede o identificador. Disciplinas sem o asterisco são consideradas obrigatórias. Sempre que uma disciplina é incluída na matriz, seu atributo \PDInline{caráter} é ajustado para \PDInline{Obrigatório} ou \PDInline{Optativo}.

\subsubsection*{Carga de optativas}
Em um determinado período do curso, várias optativas podem ser oferecidas. No exemplo dado, são ofertadas cinco disciplinas optativas no 6º trimestre. Porém, é esperado que o aluno cumpra três delas para integralizar sua carga horária.

A carga de optativas é especificada no atributo \PDInline{carga optativas}, que estabelece uma lista de número de horas, um por disciplina.

\begin{PDListing}
    carga optativas = {
        4 = {60},
        6 = {60, 60, 60},
        8 = {30, 30, 60},
    },
\end{PDListing}

A contabilização da carga horária do período é a soma das obrigatórias com a soma das cargas das optativas listadas. (Ver \Macro{PPCCargaHoraria}.)

\subsubsection*{Relação de eletivas}

\index{eletivas}
\index{disciplinas!eletivas}
Caso seja necessário, uma lista com disciplinas eletivas selecionadas pode ser criada.

\begin{PDListing}
    eletivas = {geo-hist, astro-elem},
\end{PDListing}

Essa listagem atualiza a contagem de horas de eletivas (ver \Macro{PPCCargaHoraria}).

\subsubsection*{Compensação de carga horária}\label{sec:compensacao-carga-horaria}
Para uma situação específica de um curso, o aluno poderia optar por fazer estágio ou o trabalho de conclusão de curso. Colocar ambos na grade era necessário e as duas deveriam ser indicadas como obrigatórias,  mas a contabilização de apenas um deles deveria ser feita. Neste caso, o atribuito \PDInline{carga a compensar} indica, para cada período, um valor em horas que deve ser subtraído do total.

No exemplo, \PDInline{estagio} e \PDInline{tcc} seriam disciplinas com 360\,h de carga horária cada uma. A ``área'' de estágio e TCC as inclui como obrigatórias, somando 720\,h no total, mas a carga de uma delas é compensada deixando o total de horas do curso e do período coerentes.

\begin{PDListing}
    área = {estagio-tcc}{
        nome = {Estágio ou TCC},
        8 = {estagio, tcc},  % 2x 360h
    },
    carga a compensar = {
        8 = 360,  % compensa as 360h
    }
\end{PDListing}


\section{Descrição de recursos escrita}

\subsection{Acesso às informações definidas}

\begin{Macrodef}{PPCInfo}{\MArg{nome}}{}
    Resulta no atributo de curso especificado com \MacroRef{PPCDefinaFichaDescritiva}.
\end{Macrodef}

\begin{PDExample}
    \PPCDefinaFichaDescritiva{
        centro = {Centro}{Centro de Ciências Exotéricas},
        ano = {Ano}{2025},
    }
    O curso pertence ao \PPCInfo{centro} e foi criado em em \PPCInfo{ano}.
\end{PDExample}

\begin{Macrodef}{PPCDisciplina}{\MArg{id-disciplina}\MArg{atributo}}{}
    Resulta no \Argument{atributo} da disciplina \Argument{id-disciplina} especificado com \MacroRef{PPCDefinaDisciplina}.
\end{Macrodef}


Os diversos atributos podem ser, ainda, manipulados conforme descrito no pacote \PackageName{attrtoolbox}.

\begin{Macrodef}{PPCCargaHoraria}{\OArg{nome-matriz}\MArg{nome-do-item}\OArg{texto-adicional}}{}
    A \Macro{PPCCargaHoraria} recupera o numero de horas do curso conforme definido por \MacroRef{PPCDefinaMatrizCurricular}.

    O \Argument{nome-do-item} pode ser um dos seguintes:
    \begin{itemize}
        \item \texttt{total}: número de horas total do curso;
        \item \texttt{obrigatória}: número de horas total de disciplinas obrigatórias;
        \item \texttt{obrigatória $n$}: número de horas de disciplinas obrigatórias para o período $n$;
        \item \texttt{opcional}: número de horas total de disciplinas opcionais;
        \item \texttt{opcional $n$}: número de horas de disciplinas opcionais para o período $n$;
        \item \texttt{eletiva}: número total de disciplinas definidas como eletivas.
    \end{itemize}

    O \Argument{texto-adicional}, que é opcional, é o texto que será colocado logo após o número de horas. Pode ser, por exemplo, ``\Macro{,}h'' ou ``\PDTilde horas''.

    Caso a macro seja ``estrelada'' (\Macro{PPCCargaHoraria*}), se o número de horas for igual a zero, nada é apresentado. O \Argument{texto-adicional} também é omitido neste caso.
\end{Macrodef}

\begin{Macrodef}{PPCArea}{\OArg{nome-matriz}\MArg{id-da-área}\MArg{atributo}}{}
    \Macro{PPCArea} recupera uma informação de área \Argument{id-da-área} definida em uma matriz curricular (atualmente, apenas \PDInline{nome} está disponível).

    O \Argument{nome-matriz} pode ser usado caso mais de uma matriz seja criada.
\end{Macrodef}

\subsection{Automatização de processamento}

\subsubsection*{Específicas para o PPC}

\begin{Macrodef}{PPCParaCadaPeriodo}{\OArg{nome-matriz}\MArg{macro}\MArg{código}}{}
    Esta macro permite a execução de um dado \Argument{código} para cada um dos períodos de uma matriz curricular. Em cada repetição, o número do período é armazenado em \Argument{macro}.

    O \Argument{nome-matriz} pode ser usado caso mais de uma matriz seja criada.
\end{Macrodef}

\subsection{Mais}

\subsubsection*{Genéricas}
O controle de informações usa o pacote \PDInline{attrtoolbox}\footnote{Ainda em desenvolvimento e não publicado. Projeto disponível no \href{https://github.com/jandermoreira/attrtoolbox}{Github}.} e seus recursos ficam disponibilizados.

Alguns desses recursos podem ser destacados:
\begin{itemize}
    \item \Macro{ATListForEach}
    \item \Macro{ATForEach}
    \item \Macro{ATAttributeIfExist}
    \item \Macro{ATAttributeIfEqual}
    \item \ldots
\end{itemize}


\section{Informações padronizadas}

\subsection{Informações disponíveis}

\begin{Macrodef}{PPCCapa}{}{}
    \Macro{PPCCapa} gera a capa padrão do projeto pedagógico.

    Quando usada a classe \PackageName{ppc-ufscar.cls}, ela é criada automaticamente, exceto se a opção \OptionRef{capa} for ajustada para \PDInline{false}.
\end{Macrodef}

\begin{Macrodef}{PPCDadosInstitucionais}{}{}
    \Macro{PPCDadosInstitucionais} gera a descrição padrão dos dados institucionais.

    Quando usada a classe \PackageName{ppc-ufscar.cls}, ela é criada automaticamente, exceto se a opção \OptionRef{dados institucionais} for ajustada para \PDInline{false}.
\end{Macrodef}

\begin{Macrodef}{PPCFichaDisciplina}{\MArg{id-disciplina}}{}
    \Macro{PPCFichaDisciplina} gera a ficha descritiva padrão de uma disciplina identificada por \Argument{id-disciplina}.
\end{Macrodef}


\begin{Macrodef}{PPCFichaDescritiva}{}{}
    \Macro{PPCFichaDescritiva} gera a ficha descritiva padrão para o curso.

    Quando usada a classe \PackageName{ppc-ufscar.cls}, ela é criada automaticamente, exceto se a opção \OptionRef{ficha descritiva} for ajustada para \PDInline{false}.
\end{Macrodef}


\begin{Macrodef}{PPCQuadroPeriodo}{\OArg{nome-matriz}\MArg{períodos}\OArg{legenda}}{}
    \Macro{PPCQuadroPeriodo} gera um \EnvironmentRef{quadro} para os \Argument{períodos} especificados. Os períodos são na forma de uma lista de números de período separados por vírgulas; caso a lista seja omitida, todos os períodos são gerados.

    O quadro padrão é formado por uma tabela com as disciplinas do período.

    Uma \Argument{legenda} pode ser opcionalmente especificada e seu texto será usado para a legenda do quadro. Neste texto, \Macro{periodo} pode ser usado para o número do período.
\end{Macrodef}

\begin{Macrodef}{PPCMatrizCurricular}{\OArg{nome-matriz}\AArg{opções}}{}
    O comando \Macro{PPCMatrizCurricular} gera um elemento gráfico com a matriz curricular. Caso mais de uma matriz tenha sido criada, ela pode ser especificada por \Argument{nome-matriz}.
\end{Macrodef}

Para as apresentações de dados, o ambiente \EnvironmentDef{quadro} é disponibilizado. Ele é usado de forma análoca a \Environment{figure} e \Environment{table}, por exemplo.

\subsection{Opções para as informações padronizadas}

\subsubsection*{Capa, dados institucionais e ficha da disciplina}
Quando o documento é criado usando a classe \PDInline{ppc-ufscar} (com \Macro{documentclass}\PArg{ppc-ufscar}), as folhas de capa, dos dados institucionais e da ficha descritiva são automaticamente criados. Estas opções devem ser especificadas ainda no preâmbulo do documento (antes de \PDInline{\begin{document}}).

Essas opções são definidas com \Macro{PPCConfig}.

\begin{Macrodef}{PPCConfig}{\MArg{lista-de-opções}}{}
    O comando \Macro{PPCConfig} aceita uma \Argument{lista-de-opções} separadas por vírgulas. As opções são descritas ao longo desta seção.
\end{Macrodef}

Assim, caso seja necessário suprimir algumas das páginas, as opções \OptionRef{capa}, \OptionRef{dados institucionais} e \OptionRef{ficha descritiva} podem ser usadas.

\begin{Optiondef}{capa}{\PDInline{true}|\PDInline{false}}{Inicialmente: \PDInline{true}}
    Esta opção permite suprimir a criação automática da capa do documento. A capa pode ser gerada manualmente pelo comando \MacroRef{PPCCapa}.
\end{Optiondef}

\begin{Optiondef}{dados institucionais}{\PDInline{true}|\PDInline{false}}{Inicialmente: \PDInline{true}}
    Esta opção permite suprimir a criação automática da relação de dados institucionais. A relação pode ser gerada manualmente pelo comando \MacroRef{PPCDadosInstitucionais}.
\end{Optiondef}

\begin{Optiondef}{ficha descritiva}{\PDInline{true}|\PDInline{false}}{Inicialmente: \PDInline{true}}
    Esta opção permite suprimir a criação automática da ficha descritiva do curso. A ficha pode ser gerada manualmente pelo comando \MacroRef{PPCFichaDescritiva}.
\end{Optiondef}

\begin{PDListing}
    \PPCConfig{
        capa = false,  % omite a capa (padrão é 'true')
        ficha descritiva,  % equivale a 'ficha presença = true' (padrão)
    }
        % a relação dos dados institucionais não é alterada!
\end{PDListing}

\subsubsection*{Opções para a apresentação gráfica da matriz curricular}
A aparência da apresentação pode ser modificada segundo uma série de opções. Essas opções podem ser definidas com \MacroRef{PPCGraficos}, quando têm validade no texto todo, ou especificadas nas opções de \MacroRef{PPCMatrizCurricular}, sendo aplicadas somente ao gráfico gerado.

\begin{Macrodef}{PPCGraficos}{\MArg{lista-de-opções}}{}
    O comando \Macro{PPCGraficos} aceita uma \Argument{lista-de-opções} separadas por vírgulas. As opções são descritas ao longo desta seção.
\end{Macrodef}

\begin{Optiondef}{largura caixa}{\Argument{comprimento}}{Inicialmente: \texttt{4cm}}
    Determina a largura de cada caixa que representa uma disciplina. Esse valor é uma referência, pois a figura como um todo será redimensionada.
\end{Optiondef}

\begin{Optiondef}{altura caixa}{\Argument{comprimento}}{Inicialmente: \texttt{2.5cm}}
    Determina a altura de cada caixa que representa uma disciplina. Esse valor é uma referência, pois a figura como um todo será redimensionada.
\end{Optiondef}

\begin{Optiondef}{altura nome área}{\Argument{comprimento}}{Inicialmente: \texttt{1.5cm}}
    Determina a altura que o nome da área ocupará (primeira linha da matriz curricular). Esse valor é uma referência, pois a figura como um todo será redimensionada.
\end{Optiondef}

\begin{Optiondef}{separação}{\Argument{comprimento}}{Inicialmente: \texttt{0.2cm}}
    A separação indica o espaço entre as disciplinas e as bordas. Esse valor é uma referência, pois a figura como um todo será redimensionada.
\end{Optiondef}

\begin{Optiondef}{cor fundo obrigatória}{\Argument{cor}}{Inicialmente: \texttt{blue!40}}
    Determina a cor de fundo das caixas das disciplinas obrigatórias.
\end{Optiondef}

\begin{Optiondef}{cor texto obrigatória}{\Argument{cor}}{Inicialmente: \texttt{black}}
    Determina a cor do texto nas caixas das disciplinas obrigatórias.
\end{Optiondef}

\begin{Optiondef}{cor horas obrigatória}{\Argument{cor}}{Inicialmente: \texttt{black}}
    Determina a cor do texto usado para a carga horária nas caixas das disciplinas obrigatórias.
\end{Optiondef}

\begin{Optiondef}{cor gradiente obrigatória}{\Argument{cor}}{Inicialmente: \texttt{blue!10}}
    Os espaços para as horas dentro das caixas das disciplinas possuem um fundo degradê. Esse gradiente parte da cor de fundo da caixa e termina na cor especificada para \Option{cor gradiente obrigatória}.
\end{Optiondef}

\begin{Optiondef}{cor fundo optativa}{\Argument{cor}}{Inicialmente: \texttt{teal!40}}
    Determina a cor de fundo das caixas das disciplinas optativas.
\end{Optiondef}

\begin{Optiondef}{cor texto optativa}{\Argument{cor}}{Inicialmente: \texttt{black}}
    Determina a cor do texto nas caixas das disciplinas optativas.
\end{Optiondef}

\begin{Optiondef}{cor horas optativa}{\Argument{cor}}{Inicialmente: \texttt{black}}
    Determina a cor do texto usado para a carga horária nas caixas das disciplinas optativas.
\end{Optiondef}

\begin{Optiondef}{cor gradiente optativa}{\Argument{cor}}{Inicialmente: \texttt{teal!10}}
    Os espaços para as horas dentro das caixas das disciplinas possuem um fundo degradê. Esse gradiente parte da cor de fundo da caixa e termina na cor especificada para \Option{cor gradiente optativa}.
\end{Optiondef}

\begin{Optiondef}{cor fundo área}{\Argument{cor}}{Inicialmente: \texttt{black!10}}
    As disciplinas de cada área ficam contidas em uma caixa que engloba todos os períodos. Esta é a cor de fundo desta caixa. Ela também é usada como cor de fundo para o nome das áreas.
\end{Optiondef}

\begin{Optiondef}{cor texto período}{\Argument{cor}}{Inicialmente: \texttt{blue!20!black}}
    Cor do texto usado para o número do período na primeira coluna da matriz curricular.
\end{Optiondef}

\begin{Optiondef}{cor fundo período}{\Argument{cor}}{Inicialmente: \texttt{cyan!80!black!40}}
    Cor de fundo da caixa usada para cada período na primeira coluna da matriz curricular.
\end{Optiondef}

\begin{Optiondef}{fonte texto}{\Argument{formato}}{Inicialmente: \PDInline{\sffamily\bfseries\small}}
    Comandos para especificação da fonte (família, tamanho, forma\ldots) usada nas caixas de disciplina e nome das áreas.
\end{Optiondef}

\begin{Optiondef}{fonte horas}{\Argument{formato}}{Inicialmente: \PDInline{\sffamily\bfseries\small}}
    Comandos para especificação da fonte (família, tamanho, forma\ldots) usada para as horas nas caixas de disciplina.
\end{Optiondef}

\begin{Optiondef}{fonte período}{\Argument{formato}}{Inicialmente: \PDInline{\sffamily\bfseries\LARGE}}
    Comandos para especificação da fonte (família, tamanho, forma\ldots) usada nas caixas de que indicam o período.
\end{Optiondef}

\begin{Optiondef}{mostra total disciplina}{\PDInline{true}|\PDInline{false}}{Inicialmente: \PDInline{true}}
    Se \PDInline{true}, a quantidade total de horas é apresentada em cada caixa de disciplina. Quando \PDInline{false}, o valor é omitido.
\end{Optiondef}

\begin{Optiondef}{mostra nome área}{\PDInline{true}|\PDInline{false}}{Inicialmente: \PDInline{true}}
    Quando ajustada para \PDInline{true}, essa opção implica na exibição das caixas com os nomes das áreas. Toda a primeira linha da matriz curricular é ocultada se ajustada para \PDInline{false}.
\end{Optiondef}

\begin{Optiondef}{mostra coluna períodos}{\PDInline{true}|\PDInline{false}}{Inicialmente: \PDInline{true}}
    Quando ajustada para \PDInline{true}, essa opção implica na exibição da coluna com os períodos e horas, que é ocultada se ajustada para \PDInline{false}.
\end{Optiondef}

\begin{Optiondef}{visão do aluno}{\PDInline{true}|\PDInline{false}}{Inicialmente: \PDInline{false}}
    Na apresentação regular da matriz curricular, todas as disciplinas optativas disponíveis para o aluno são apresentadas, mesmo que ele tenha que cumprir uma quantidade menor delas.

    Na ``visão do aluno'', as áreas não são apresentadas e as optativas são representadas de forma genérica, indicando a quantidade e as cargas horárias de cada uma.
\end{Optiondef}


% \section{Recursos de compilação condicional}
% \index{Overleaf}%
% \index{timeout@\textit{timeout}}%
% Os projetos que usam a classe \PackageName{ppc-ufscar} tendem usar uma quantidade relativamente alta de recursos do \LaTeX\ e também aumentam o tempo de compilação. Em particular, quando usado no Overleaf\footnote{\url{https://www.overleaf.com}.}, é comum que a compilação falhe por \textit{timeout} (reduzido para 20~segundos para as contas grátis em 2024).


\section{Competências}
\textit{A fazer\ldots}


\section{Objetivos de Desenvolvimento Sustentável (ODS)}
\textit{A fazer\ldots}


\section{Ainda por fazer}
Esta é uma \textit{lista de desejos} que podem ou não ser concretizados no futuro:

\begin{itemize}
    \item Substituição do texto usado nas bibliografias das disciplinas por uma lista de rótulos \BibTeX\ para formatação automática segundo a ABNT.
    \item Geração de grade de disciplinas com conexões representando pré-requisitos.
    \item Pensar em uma solução melhor para o caso da compensação de horas (pág.~\pageref{sec:compensacao-carga-horaria}).
\end{itemize}

\printindex

\end{document}
